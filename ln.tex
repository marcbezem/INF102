\documentclass[handout]{beamer}
%\documentclass[slides]{beamer}
% Vary the color applet  (try out your own if you like)
%\colorlet{structure}{red!20!black}
%\beamertemplateshadingbackground{yellow!20}{white}
%\usepackage{beamerthemeshadow}
%\usepackage[utf8x]{inputenc} CONFLICT!
%\usepackage[english,norsk,nynorsk]{babel}
\usepackage{marvosym}
\usepackage[all]{xy}
%\usepackage{amsmath}
\usepackage{multicol}

\setbeamertemplate{navigation symbols}{}
\setbeamertemplate{footline}[frame number]
\usetheme{Montpellier}

\newcommand{\va}{\mathit{variable}}
\newcommand{\Fa}{\mathrm{F}}
\newcommand{\TEXT}{{\mathrm{TEXT}}}
\newcommand{\INC}{{\mbox{INC}\,}}
\newcommand{\DEC}{{\mbox{DEC}\,}}
\newcommand{\CLR}{{\mbox{CLR}\,}}
\newcommand{\JND}{{\mbox{JND}\,}}
\newcommand{\JZ}{{\mbox{JZ}\,}}
\newcommand{\JMPp}{{\mbox{JMP}_{\!+}\,}}
\newcommand{\JMPn}{{\mbox{JMP}_{\!-}\,}}
\newcommand{\CONT}{{\mbox{CONT}}}
\newcommand{\Nat}{\mathbb{N}}
\newcommand{\Int}{\mathbb{Z}}
\newcommand{\Rat}{\mathbb{Q}}
\newcommand{\Rea}{\mathbb{R}}
\newcommand{\cS}{\mathcal{S}}
\newcommand{\cD}{\mathcal{D}}
\newcommand{\cT}{\mathcal{T}}
\newcommand{\cL}{\mathcal{L}}
\newcommand{\cB}{\mathcal{B}}
\newcommand{\cF}{\mathcal{F}}
\newcommand{\cN}{\mathcal{N}}
\newcommand{\cP}{\mathcal{P}}
\newcommand{\cA}{\mathcal{A}}
\newcommand{\ra}{\rightarrow}
\newcommand{\la}{\leftarrow}
\newcommand{\set}[1]{\{#1\}}
\newcommand{\seg}[2]{[#1.\,.#2]}
\newcommand{\pair}[2]{\langle #1,#2\rangle}
\newcommand{\der}[2]{\frac{\partial#1}{\partial#2}}
\newcommand{\plusD}[1]{#1+\Delta{#1}}
\newcommand{\dD}[1]{\Delta{#1}}
\newcommand{\range}{\mathit{range}}
\newcommand{\ovx}{\overline{x}}
\newcommand{\ovy}{\overline{y}}
\newcommand{\ovl}[1]{\overline{#1}}
\newcommand{\ul}[1]{\underline{#1}}
\newcommand{\Sigmeps}{\Sigma_\epsilon}
\newcommand{\Gammeps}{\Gamma_\epsilon}
\newcommand{\ea}{\epsilon_1}
\newcommand{\eb}{\epsilon_2}
\newcommand{\ec}{\epsilon_3}
\newcommand{\ba}{\frac{100}{2}b_1}
\newcommand{\bb}{\frac{10}{2}b_2}
\newcommand{\bc}{\frac{1}{2}b_3}
\newcommand{\binomsq}[2]{\left[\begin{array}{c}#1\\#2\end{array}\right]}
\newcommand{\wrd}[1]{\mathbf{#1}}
\newcommand{\myiff}{\Leftrightarrow}
\newcommand{\myimp}{\Rightarrow}
\newcommand{\To}{\Rightarrow}
\newcommand{\Trt}{\stackrel{*}{\Rightarrow}}
\newcommand{\ToG}{\Rightarrow_G}

\newcommand{\git}{https://github.com/marcbezem/INF102/blob/master}

\title[INF102]%
{INF102\\Algorithms, Data Structures and Programming}
\author[Marc Bezem]{%
  Marc~Bezem\inst{1} %\and
  }
\institute[University of Bergen]{
  \inst{1}%
  Department of Informatics\\
  University of Bergen
  %\and
  }
\date[INF102]{Fall 2017}

\subject{Algorithms and Data Structures}

%\pgfdeclaremask{fsu}{fsu_logo_ybkgrd}
%\pgfdeclareimage[mask=fsu,width=1cm]{fsu-logo}{fsu_logo_ybkgrd}

%\logo{\vbox{\vskip0.1cm\hbox{\pgfuseimage{fsu-logo}}}}

\begin{document}

  \frame
  {
    \titlepage
  }

\section{Introduction}

 \frame
  {
    
    \frametitle{INF102, practical stuff}\label{practicalstuff}

\begin{itemize}[<+->]
\item Lecturer: Marc Bezem; Team: see homepage
\item  Homepage:
\href{https://mitt.uib.no/courses/7191}%
{\color{red}{INF102}} (requires login)
\item Also: 
\href{https://github.com/marcbezem/INF102}%
{\color{red}{INF102 on GitHub}}
\item Tentative
\href{\git/lecturenotes/plan.pdf}%
{\color{red}{schedule}}
\item Textbook:
\href{http://algs4.cs.princeton.edu/home/}{\color{red}{Algorithms, 4th edition}}
%, R. Sedgewick and K. Wayne, Pearson, 2011
\item Prerequisites: INF100 + 101 ($\approx$ Ch.\ 1.1 + 1.2)
\item  Syllabus (pensum): Ch.\ 1.3 -- 1.5, Ch. 2 -- 4
\item Three compulsory exercises, must be passed
\item Digital exam (Inspera)
\href{http://www.uib.no/student/108376/eksamensdatoer-det-matematisk-naturvitenskapelige-fakultet-høsten-2017-exam-dates}%
{\color{red}{06.12.2017}}
\item Old exams: 
\href{https://github.com/marcbezem/INF102/tree/master/oldexams}%
{\color{red}{2004--2017}}
\item \hyperlink{contentstable}{\color{red}{Table of Contents of these slides}}
\end{itemize}
}

 \frame
  {
    
    \frametitle{Resources}
\begin{itemize}[<+->]
\item Good textbook, USA-style: many pages, exercises etc.
\item Average speed must be ca 50 pages p/w
\item Lectures (ca 24) focus on the essentials
\item Slides (ca 120, dense!) summarize the lectures
\item Prepare yourself by reading in advance
\item Workshops: help with \href{\git/lecturenotes/plan.pdf}%
{\color{red}{selected}} exercises
\item Test yourself by trying some exercises in advance
\item If you can do the exercises (incl.\ compulsory), you are fine
\item Review of exercises on Tuesday morning
\end{itemize}
}

\section{Ch.1.3 Bags, Queues and Stacks}

\frame
  {
    
    \frametitle{Generic Bags, Queues and Stacks}

\begin{itemize}[<+->]
\item Generic programming in Java, example: 
 \href{\git/programs/oddsAndEnds/PolyPair.java}%
{\color{red}{PolyPair.java}}
\item Bag, Queue and Stack are generic, iterable collections
\item All three support adding an element
\item Queue and Stack support removing an element (if any)
\item FIFO Queue (en/dequeue), LIFO Stack (push/pop)
\item Stack: INF101; Bag is a Stack without {\tt pop}
\item APIs include: {\tt boolean isEmpty()} and {\tt int size()}
\end{itemize}
}

\frame
  {
    
    \frametitle{Implementations and one application}

\begin{itemize}[<+->]

\item Implementation of Stack:
\href{\git/programs/fundamentals/bagsQueuesAndStacks/LinkedList_Stack.java}%
{\color{red}{LinkedList\_Stack.java}}
\item Implementation of Queue:
\href{\git/programs/fundamentals/bagsQueuesAndStacks/LinkedList_Queue.java}%
{\color{red}{LinkedList\_Queue.java}}
\item Dijkstra's Two-Stack Expression Evaluation
\item Example:  ( 1 + (  ( 2 + 3 )  *  ( 4 * 5 ) ) )
\item Now you learned something about compilers!
\item \href{http://algs4.cs.princeton.edu/lectures/13DemoDijkstraTwoStack.mov}{\color{red}{Movie}}
\end{itemize}
}


\frame
  {
    
    \frametitle{Resizing Arrays}

\begin{itemize}[<+->]
\item Arrays have direct access, but have fixed size
\item Linked lists have flexible size, but no direct access
\item Best of both: use new, resized arrays, \emph{wisely}:
  \begin{itemize}[<+->]
  \item double size when the array becomes overfull
  \item halve size when the array becomes quarter full
  \end{itemize}  
\item Resizing takes time and space proportional to size
\item Not too seldom (correctness), not too often (efficiency)
\item Later: we retain \emph{constant time direct access}
\item Later: add operation in \emph{constant time on average}
\item Once we have understood resizing arrays: \href{http://docs.oracle.com/javase/7/docs/api/java/util/ArrayList.html}%
{\color{red}{ArrayList}}
\end{itemize}    
}

\frame
  {
    
    \frametitle{Implementations}

\begin{itemize}[<+->]
\item 
\href{\git/programs/fundamentals/bagsQueuesAndStacks/ResizingArray_Stack.java}%
{\color{red}{ResizingArray\_Stack.java}}
\item Arrays give direct access, resizing at reasonable cost
\item
\href{\git/programs/fundamentals/bagsQueuesAndStacks/LinkedList_Stack.java}%
{\color{red}{LinkedList\_Stack.java}}
\item No fixed size, but indirect access incurs a cost
\item Pointers take space and dereferencing takes time
\item Programming with pointers: make a picture
\end{itemize}    
}

\section{Ch.1.4 Analysis of Algorithms}

\frame
  {
    
    \frametitle{Computation time and memory space}

\begin{itemize}[<+->]
\item Two central questions:
  \begin{itemize}
  \item How long will my program take?
  \item Will there be enough memory?
  \end{itemize}
\item Example: 
\href{http://algs4.cs.princeton.edu/14analysis/ThreeSum.java.html}%
{\color{red}{ThreeSum.java}}
\item Inner loop (here {\tt a[i]+a[j]+a[k]==0}) is important
\item Sorting helps: \href{\git/programs/oddsAndEnds/ThreeSumOptimized.java}%
{\color{red}{ThreeSumOptimized.java}}
\item Run some experiments: {\tt 1Kints.txt, 2Kints.txt, ...}
%\item Observe the `doubling factor'
\end{itemize}  
}

\frame
  {
    
    \frametitle{Methods of Algorithm Analysis}

\begin{itemize}[<+->]

\item Empirical:
  \begin{itemize}
  \item Run program with randomized inputs, measuring time \& space
  \item Run program repeatedly, varying (doubling) the input size
  \item Measuring time: \href{http://algs4.cs.princeton.edu/code/javadoc/edu/princeton/cs/algs4/Stopwatch.html}%
  {\color{red}{StopWatch}}
  \item Plot, or log-log plot and 
\href{http://www.alcula.com/calculators/statistics/linear-regression/}{\color{red}{linear regression}}  
  \end{itemize}
\item Theoretical:
  \begin{itemize}
  \item Define a cost model by abstraction (e.g., array accesses, comparisons, operations)
  \item Try to count/estimate/average this cost as function of the input (size)
  \item Use $O(f(n))$ (MNF130) and $f(n)\sim g(n)$ (see next slide)
  \end{itemize}
\end{itemize}  
}

\frame
  {
    
    \frametitle{Orders og Growth, Big Oh and $\sim$}

\begin{itemize}[<+->]

%\item Q: `wins for large $n$' uhh???
%\item A: Big Oh, and $\sim$ will clear this up
\item Big Oh and $\sim$ aim to capture `order of growth'
\item Costs are positive quantities, so $f,g,... : \Nat \to \Rea^+$
\item MNF130:  $f(n)$ is $O(g(n))$ if there exist 
$c\in\Rea^+,~N\in\Nat$ such that $f(n) \leq c g(n)$ for all $n\geq N$
(that is, for $n$ large enough)
\item Example: $n^2$ and even $99n^3$ are $O(n^3)$, but  $n^3$ is not $O(n^{2.9})$
\item INF102:  $f(n)\sim g(n)$ if $1 = \lim_{n\to\infty} (f(n) / g(n))$
\item If $f(n)\sim g(n)$, then $f(n)$ is $O(g(n))$ and $g(n)$ is $O(f(n))$
\item Not conversely: Big Oh disregards constant factors, $\sim$ not
\item Factor $c$ hidden in Big Oh is important in practice
\item Bound $N$ is important if it is large
\end{itemize}  
}

\frame
  {
    
    \frametitle{Important orders of growth}

\begin{tabular}{l | l}
order of growth as function of $n$ & value for $n=20$ \\\hline\hline
constant: $c$, meaning $f(n) = c$ for all $n$ & $c$ sec\\\hline
linear: $n$                                                               & $20$ sec \\\hline
linearithmetic: $n \log n$                                    & $26$ sec \\\hline
quadratic: $n^2$                                                   & $400$ sec\\\hline
cubic: $n^3$                                                           & $8000$ sec\\\hline
exponential: $2^n$                                               & $1048576$ sec\\\hline
general form: $a n^b (\log n)^c$ & $a\cdot 20^b \cdot(1.3)^c$ sec

\end{tabular}  
}

\frame
  {
    
    \frametitle{ThreeSum, theoretically}

\begin{itemize}[<+->]

\item Number of different picks of triples: $g(n) = n(n-1)(n-2)/6$
\item Inner loop {\tt a[i]+a[j]+a[k]==0} executed $g(n)$ times
\item $f(n) = n^3 / 6 - n^2 / 2 + n / 3$
\item Cubic term $n^3 / 6$ wins for large $n$: $f(n) \sim n^3 / 6$        
\item Cost model \# array accesses: ${} \sim n^3 / 2$
\item Cost array access $t$ sec: total time  ${} \sim  t * n^3 / 2$ sec
\item Cost models are (necessary) simplifications!
\end{itemize}  
}


\frame
  {
    
    \frametitle{ThreeSum, empirically}

\begin{itemize}[<+->]

\item Input sizes 1K, 2K, 4K, 8K take time 0.1, 0.8, 6.4, 51.1 sec
\item The log's are 3, 3.3, 3.6, 3.9 and -1, -0.1, 0.8, 1.71
\item See the plots in \href{\git/lecturenotes/plots_1_4.ods}%
{\color{red}{plot sheet}}
\item Linear regression gives $y \approx 3x -10$
\item $\log(f(n))= 3\log(n) -10$ iff 
         $$f(n)=10^{\log(f(n))} = 10^{3\log(n) - 10} =n^{3}*10^{-10}$$
         
\item Conclusion: cubic in the input size, with constant $\approx 10^{-10}$
\item No surprise: see the 3-nested loop in
\href{http://algs4.cs.princeton.edu/14analysis/ThreeSum.java.html}%
{\color{red}{ThreeSum.java}}
\item Strong dependence on input can be a problem
\item Constant $10^{-10}$ depends on computer, exponent 3 does not

\end{itemize}  
}


\frame
  {
    
    \frametitle{Worst case, average case, amortized cost}

\begin{itemize}[<+->]

\item Worst case: guaranteed, independent of input; Examples:
  \begin{itemize}
  \item Linked list implementations of Stack, Queue and Bag: all operations
  take constant time in the worst case
  \item Resizing array implementations of Stack, Queue and Bag: adding and deleting
  take linear time in the worst case (easy)
  \end{itemize}
\item Average case: not guaranteed, dependent of input \emph{distribution}
\item Amortized: worst-case cost  \emph{per operation}. 
E.g., each 10-th operation has cost $\leq 21$, all others cost $1$, 
amortized $\leq 3$ p/o.
\item Resizing arrays: adding and deleting
take constant time \emph{per operation} in the worst case (proof is difficult)
\item Special case of resizing array that is only growing: 
         1(2)2(4)34(8)5678(16)9 ... 16(32) ..., with $(n)$ the new size.
         Risizing to $(n)$ costs $2n$ array accesses, so in total
         (1+4)+(1+8)+(2+16)+(4+32)+(8+64) ..., so $9$ p/push.
                 
\end{itemize}  
}

\begin{frame}

    \frametitle{Remarks and Pitfalls}

\begin{itemize}[<+->]
\item Theoretical approach:
  \begin{itemize}
  \item Wrong cost model
  \item JVM optimization can obscure the exponent
  \item Caching can have large impact on memory access
  \item Large constant factor in Big Oh
  \item Worst case can be easy, average case difficult
  \end{itemize}
\item Empirical approach:
  \begin{itemize}
  \item The focus is on run time (using space costs time)
  \item Dependence on input, randomization does not always help
  \item Machine/platform dependence
  \item Linear regression not good for, e.g., $O(n^2 \log n)$
  \end{itemize}
\end{itemize}     

\end{frame}

\frame
  {
    
    \frametitle{Exercise}

Aim: better understand the empirical method.

\begin{enumerate}[<+->]

\item Let input sizes 1, 2, 4, 6, 8K take 2, 7.9, 32, 72,  129 sec
\item Make a plot such as in \href{\git/lecturenotes/plots_1_4.ods}%
{\color{red}{plot sheet}} (download)
\item Compute the log's of the input sizes and of the run times
and make the log-log plot such as in \href{\git/lecturenotes/plots_1_4.ods}%
{\color{red}{plot sheet}} (second plot)
\item Estimate $a$ and $b$ such that the log-log plot is $y \approx ax - b$
\item Estimate $a$ and $b$ through
\href{http://www.alcula.com/calculators/statistics/linear-regression/}{\color{red}{linear regression}}, compare with 4.
\item Find $f(n)$ given that $\log(f(n))= a\log(n) - b$. Surprised?
\end{enumerate}  
In cases where the run time mostly depends on the size $n$ of the input
and not on the input itself, the function $f$ is a reasonable 
(polynomial) estimation of the run time.
}

\begin{frame}

    \frametitle{Logarithms and Exponents Cheat Sheet}

\begin{itemize}[<+->]
\item Definition: $\log_x z = y$ iff $x^y = z$ for $x>0$
\item Base of logarithm: the $x$ in $\log_x$
\item Inverses: $x^{\log_x y} = y$ and $\log_x x^y = y$
\item Exponent, laws: $x^{(y+z)} = x^y x^z$, $x^{(yz)} = (x^y)^z$
\item Logarithm, laws: $\log_x (yz) = \log_x y + \log_x z$, $\log_x z = \log_x y \log_y z$
\item Various bases: $\log_2 = \lg$, $\log_e = \ln$, $\log_{10} = \log$
\item Double exponent: e.g.\ $2^{(2^n)}$ (not used in INF102)
\item Double logarithm: $\log(\log n)$ (not used in INF102)
\end{itemize}     

\end{frame}

\section{Ch.1.5 Case Study: Union-Find}

\frame
  {
    
    \frametitle{Staying Connected}

\begin{itemize}[<+->]
\item We want efficient algorithms and datastructures 
for testing whether two objects are `connected' (e.g., in networks)
\item We assume connectedness to be an equivalence
\item MNF130: relation $E\subseteq V\times V$ is an \emph{equivalence} if
  \begin{itemize}
  \item $E$ is \emph{reflexive}: $\forall x\in V.~ E(x,x)$
  \item $E$ is \emph{symmetic}: $\forall x,y\in V.~ E(x,y) \to E(y,x)$
  \item $E$ is \emph{transitive}: $\forall x,y,z\in V.~ E(x,y)\land E(y,z) \to E(x,z)$
  \end{itemize}
\item Dynamic connectivity means (here) that $E$ can grow %(not: shrink)
%\item Example: if the `Bergensbanen' is broken, Oslo and Bergen are no longer connected by rail
\item Relationship with paths in graphs, (connected) components (MNF130): 
nodes are connected if there is a path between them
\item Input: $N$ and  pairs in $V=\set{0,\ldots,N{-}1}$ defining $E$
\item Challenge: efficient {\tt boolean connected(int p, int q)} 
\end{itemize}  
}

\frame
  {
    
    \frametitle{Example}

\begin{itemize}[<+->]

\item Example (\url{algs4-data/tinyUF.txt}) : $N=10$
\item Nodes $0,1,2,3,4,5,6,7,8,9$
\item Edges: $4{-}3$, $3{-}8$,  $6{-}5$, $9{-}4$, $2{-}1$, $8{-}9$, $5{-}0$, ...
\item Linear space: don't add pairs that are already connected!
\item Q: what are the costs of storing all pairs that are connected, space and time?
\item See: \href{https://bitbucket.org/algoritmevisualisering/algoritmevisualisering}%
{{\color{red}{algoritmevisualisering}}} by Ragnhild \AA lvik, Kristian Rosland, Knut Anders Stokke
\end{itemize}  
}

\frame
  {
    
    \frametitle{Union-Find}

\begin{itemize}[<+->]
\item Find, idea: every component has one element as its identifier, 
 {\tt int  find(int n)} computes this identifier
\item Union, idea: for any new pair $n~m$ that are not already connected, 
{\tt union(int n, int m)} takes the union of the two components, ensuring
{\tt find(n) == find(m)}
\item API: \href{http://algs4.cs.princeton.edu/code/javadoc}{{\color{red}{UF}}};
Cost model: number of array accesses
\item Implementations:
  \begin{itemize}
  \item \href{\git/programs/fundamentals/unionFind/SlowUF.java}%
{\color{red}{SlowUF.java}}: {\tt id[p]} identifier of {\tt p}\\
{\tt find()}$\sim1$, {\tt union()}$\sim 2$ or between $n{+}3$ and $2n{+}1$ (!)
  \item \href{\git/programs/fundamentals/unionFind/FastUF.java}%
{\color{red}{FastUF.java}}: {\tt int[] id} pointers, {\tt id[p]==p}:  identifier\\
{\tt find()}  $\sim 1{+}2d$, {\tt union()} $\sim 1{+}$ two {\tt find()}'s
  \item \href{\git/programs/fundamentals/unionFind/WeightedUF.java}%
{\color{red}{WeightedUF.java}}: {\tt int[] id} pointers, {\tt int[] sz} subtree sizes\\
{\tt find()} and {\tt union()} both $\sim \lg n$
  \end{itemize}
\end{itemize}  
}

\frame
  {
    
    \frametitle{Trees (cf.\ MNF130) and WeightedUF}
A (rooted) \emph{tree} consist of \emph{nodes} (also called \emph{vertices})
one of which is called the \emph{root} $r$. 
Every node $n$ is connected by an \emph{edge} to zero 
or more other nodes, called the \emph{children} of the \emph{parent }$n$.
Moreover, each node $n\neq r$ has a unique parent in a tree.
Trees are naturally depicted in levels starting with the root at level 0,
then the level 1 of the children of the root, level 2 of the children
of the children of the root, and so on. The level of a node is also called its
\emph{depth}. In a finite tree there is always a highest level (maximum depth)
and this is called the \emph{heigth} of the tree.
\begin{itemize}[<+->]
\item WeightedUF: height of subtree of size $k$ is  at most $\lg k$ 
(proof by induction on blackboard)
\item Ultimate improvement of UF (almost $O(1)$, amortized): 
path-compression (sketch on bb)
\end{itemize} 
}

\section{Ch.2.1 Elementary Sorts}

\frame{
    \frametitle{Sorting}

\begin{itemize}[<+->]
\item Sorting: putting objects in a certain order
\item MNF130: relation $R\subseteq V\times V$ is a \emph{total order(ing)} if
  \begin{enumerate}
  \item $R$ is \emph{reflexive}: $\forall x\in V.~ R(x,x)$
  \item $R$ is \emph{transitive}: $\forall x,y,z\in V.~ R(x,y)\land R(y,z) \to R(x,z)$
    \item $R$ is \emph{antisymmetic}: $\forall x,y\in V.~ R(x,y) \land R(y,x) \to x=y$
    \item $R$ is \emph{total}: $\forall x,y\in V.~ R(x,y) \lor R(y,x)$
 \end{enumerate}
\item Natural orderings: 
  \begin{itemize}
  \item Numbers of any type: ordinary $\leq$ and $\geq$
  \item Strings: lexicographic
  \item Objects of a Comparable type: {\tt v.compareTo(w) <= 0} 
  \end{itemize}
\end{itemize}  
}


\begin{frame}[fragile]
    \frametitle{Sorting and searching: linear vs  binary search}
\small
\begin{verbatim}
int linearSearch(Comparable key, Comparable[] a){
  for (int i=0; i < a.length; i++){ 
    if (key.compareTo(a[i])==0) {return i;}
  }
  return -1; // key not in array: O(a.length)
}    
\end{verbatim}
\begin{verbatim}
int binarySearch(Comparable key, Comparable[] a) {
  int lo=0; int hi=a.length-1; int mid; int test
  while (lo <= hi){
    mid = (lo+hi)/2; test = key.compareTo(a[mid]);
    if (test == 0) {return mid;}
    if (test < 0) { hi = mid-1;} else {lo = mid+1;}
  }
  return -1; // key not in SORTED array: O(lg(a.length))
}    
\end{verbatim}
\end{frame}


\frame{
    \frametitle{Sorting (ctnd)}

\begin{itemize}[<+->]
\item Elementary sorts: 
  \begin{enumerate}
  \item Bubble sort (like gas bubbles in sparkling water)
  \item Selection sort (iterated selection of minima)
  \item Insertion sort (iterated insertion of elements) 
  \item Shell sort (Shell's refinement of insertion sort)
 \end{enumerate}
\item Bubble sort: 
\href{\git/programs/sorting/elementarySorts/ExampleSort.java}%
{\color{red}{ExampleSort.java}}
\item Certification: {\tt assert isSorted(a)} in {\tt main()}\\
(no guarantee against modifying the array, but {\tt exch()} is safe)
\item Costmodel(s): number of {\tt less()}'s and of {\tt exch()}'s\\
(or array accesses; discuss pointer vs.\ object)
%\item Costmodel 2: number of array accesses
%\item Pitfalls: cache misses, expensive {\tt v.compareTo(w) < 0}
\item Why studying sorting? ({\tt java.util.Arrays.sort()})
\item Comparing sorting algorithms:
\href{\git/programs/sorting/SortCompare.java}%
{\color{red}{SortCompare.java}}
\end{itemize}  
}

\begin{frame}[fragile]
    \frametitle{Selection Sort}

\begin{itemize}[<+->]
\item Bubble sort: $\sim n^2 / 2$ compares, $0\leq \text{exchanges} \leq n^2 / 2$
\item Selection sort:
  \begin{itemize}
  \item Find index of a minimum in {\tt a[0..n-1]}, exchange with {\tt a[0]}
  \item Find index of a minimum in {\tt a[1..n-1]}, exchange with {\tt a[1]}
  \item ... until {\tt n-2}
  \end{itemize}
\item Selection sort: $\sim n^2 / 2$ compares, $0\leq \text{exchanges} \leq n-1$ (!)
\end{itemize}  
\begin{verbatim}
public  static void sort(Comparable[] a) {
 int N = a.length;
 for (int i=0; i<N-1; i++){
   int min=i;
   for (int j=i+1; j<N; j++) if less(a[j],a[min])) min=j;
   if (i != min) exch(a,i,min);
 }
}    
\end{verbatim}  
\end{frame}


\begin{frame}[fragile]
    \frametitle{Insertion sort}

\begin{itemize}[<+->]
\item Insertion sort: 
  \begin{itemize}
  \item Insert {\tt a[1]} on its correct place in (sorted) {\tt a[0..0]}
  \item Insert {\tt a[2]} on its correct place in (sorted) {\tt a[0..1]}
  \item ... until {\tt a[n-1]}
  \end{itemize}
\item Very good for partially sorted arrays, costs:
  \begin{itemize}
  \item Best case: $n{-}1$ compares and 0 exchanges 
  \item Worst case: $\sim n^2 / 2$ compares and exchanges
  \item Average case: $\sim n^2 / 4$ compares and exchanges (distinct keys)
  \end{itemize}
\end{itemize}   
\begin{verbatim}
public  static void sort(Comparable[] a) {
 int N = a.length;
 for (int i=1; i<N; i++){
  for (int j=i; j>0 && less(a[j],a[j-1]); j--)
   exch(a,j,j-1);
 }
}    
\end{verbatim}  
\end{frame}


\begin{frame}[fragile]
    \frametitle{Shell sort}

\begin{itemize}[<+->]
\item Insertion sort: 
  \begin{itemize}
  \item Very good for partially sorted arrays
  \item Slow due to one-step transport {\tt exch(a,j,j-1)}
  \item Why not larger steps {\tt exch(a,j,j-h)} ?
  \end{itemize}
\item Idea: presort {\tt a[i],a[i+h],a[i+2h],...} for {\tt i = 0..h-1}% in {\tt [0..h)}
\end{itemize}   
\begin{verbatim}
public static void hsort(int h, Comparable[] a) {
 int N = a.length;
 for (int i=h; i<N; i++)
  for (int j=i; j-h>=0 && less(a[j],a[j-h]); j-=h)
    exch(a,j,j-h);  
}
\end{verbatim}
\begin{itemize}[<+->]
\item Insertion sort: {\tt hsort(1,a)}
\item Shell sort: e.g., {\tt hsort(10,a); hsort(1,a)}
\end{itemize}     
\end{frame}

\begin{frame}[fragile]
    \frametitle{Shell sort (ctnd)}

\begin{itemize}[<+->]
\item {\tt hsort(10,a); hsort(1,a)} faster than just {\tt hsort(1,a)} !
\item Q: How is this possible?
\item A: {\tt hsort(10,a)} transports items in steps of 10, 
which would be done by {\tt hsort(1,a)} in 10 steps of 1.
\item What about {\tt hsort(100,a); hsort(10,a); hsort(1,a)}?
\item To be expected: depends on the length N of the array
\item The run-time analysis of Shell sort  is very difficult
\item Best practice: $h=N/3,N/9,\ldots,364,121,40,13,4,1$
\item Example: \href{https://bitbucket.org/algoritmevisualisering/algoritmevisualisering}%
{{\color{red}{algoritmevisualisering}}} by Ragnhild \AA lvik, Kristian Rosland, Knut Anders Stokke
\end{itemize}     
\end{frame}


%\end{document}

\section{Ch.2.2 Mergesort}

\begin{frame}[fragile]
    \frametitle{Mergesort}

\begin{itemize}[<+->]
\item Top-down (recursive) algorithm:
  \begin{itemize}[<+->]
  \item Mergesort left half, mergesort right half
  \item Merge the results (example: 2468,1357)
  \end{itemize}     
\item Using an auxiliary array:
\href{\git/programs/sorting/mergeSort/TopDownMergeSort.java}%
{\color{red}{TopDownMergeSort.java}},~
\href{http://algs4.cs.princeton.edu/lectures/22DemoMerge.mov}{\color{red}{Movie}}
\item Bottom-up algorithm (16 elements):
  \begin{itemize}[<+->]
  \item Merge {\tt a[0],a[1]}, merge {\tt a[2],a[3]}, merge {\tt a[4],a[5]}, ...
  \item Merge {\tt a[0..1],a[2..3]}, merge {\tt a[4..5],a[6..7]},  ...
  \item Merge {\tt a[0..3],a[4..7]}, merge {\tt a[8..11],a[12..15]}
  \item Merge {\tt a[0..7],a[8..15]}, done!
  \end{itemize}     
\item Also using an auxiliary array:
\href{\git/programs/sorting/mergeSort/BottomUpMergeSort.java}%
{\color{red}{BottomUpMergeSort.java}}
\end{itemize}     
\end{frame}

\begin{frame}[fragile]
    \frametitle{Run-time and memory use of mergesort}

\begin{itemize}[<+->]
\item Mergesort uses between $\sim (N/2) \lg N$ and $\sim N \lg N$ compares.
Proof on bb. Important formula ($N=2^n$):
$$2C(2^{n-1}) + 2^{n-1} \leq C(2^{n}) \leq 2C(2^{n-1}) + 2^{n}$$
\item Mergesort uses at most $\sim 6N \lg N$ array accesses
\item Mergesort uses $\sim 2N$ space (plus some var's)
\item Q: How fast can compare-based sorting of $N$ distinct keys be?
\item A: $\lg N! \sim N\lg N$; Proof in book and on bb. Keywords: 
binary \emph{compare tree}, inner nodes for each {\tt compare(a[i],a[j])}, 
permutations in the leaves,
$$ N! = \text{number of permutations}  
     \leq \text{number of leaves} \leq 2^\text{height of tree} $$
\end{itemize}     
\end{frame}


\section{Ch.2.3 Quicksort}

\begin{frame}
    \frametitle{Quicksort}

\begin{itemize}[<+->]
\item Top-down (recursive) algorithm:
  \begin{itemize}[<+->]
  \item Choose a (pivot)  value $v$ in the array
  \item Partition the array in non-empty parts $\leq v$ and $\geq v$
  \item Quicksort the two parts
  \end{itemize}     
\item Examples: \href{https://bitbucket.org/algoritmevisualisering/algoritmevisualisering}%
{{\color{red}{algoritmevisualisering}}} by Ragnhild \AA lvik, Kristian Rosland, Knut Anders Stokke
\item Pros: in-place, average computation time $O(n\log n)$
\item Cons: stack space for recursion, worst-case $O(n^2)$, not stable
\item Implementation:
\href{\git/programs/sorting/quickSort/QuickSort.java}%
{\color{red}{QuickSort.java}}
\item BTW: \href{http://www.envisage-project.eu/proving-android-java-and-python-sorting-algorithm-is-broken-and-how-to-fix-it/}%
{\color{red}{Bug in java.util.Arrays.sort}}
\end{itemize}     
\end{frame}

\begin{frame}
    \frametitle{Quicksort, details}

\begin{itemize}[<+->]

\item Subtleties in {\tt Quicksort.sort()}: shuffling protects against worst-case behaviour
\item Termination of recursive {\tt quicksort()}
\item Subtleties in {\tt partition()}:
  \begin{itemize}[<+->]
  \item Invariants {\tt l<=h} in the two inner loops
  \item Postcondition after the two inner loops
  \item Invariant of the {\tt for(;;)} loop
  \item Termination of the {\tt for(;;)} loop
  \item There are some variations that are also correct
  \end{itemize}    

\end{itemize}     
\end{frame}

\begin{frame}
    \frametitle{Run-time and memory use of quicksort}

\begin{itemize}[<+->]
\item Compare Quicksort to other sorts $(n=10^2, 10^3,\ldots)$
\item Quicksort: time $O(n^2)$ if pivot is always smallest (or largest)
\item Randomization: choose pivot randomly, or shuffle array
\item If all keys are distinct and randomization is perfect, then quicksort
uses on average $\sim 2n\ln n$ compares and $\sim (n/3) \ln n$
exchanges (proofs in book, complicated)
\item Relevant improvements:
  \begin{itemize}[<+->]
  \item Cut-off to insertion sort for sizes $\leq 15$ (ca.)
  \item Median-of-three pivot
  \item Taking advantage of duplicate keys (3-way partitioning)
  \end{itemize}
\item Quicksort is generally quite good
\item In special situations other sorts are better (e.g., countsort)
\end{itemize}
\end{frame}

\section{Ch.2.4 Priority Queues}

\begin{frame}
    \frametitle{Priority Queues}

\begin{itemize}[<+->]
\item Aim: collecting and processing items having keys
\item Examples of keys: time-stamp, price-tag, priority-tag
\item Assume: keys are ordered
\item Reasonable: processing currently highest (or lowest)
\item Special cases: items time-stamped when added
  \begin{itemize}[<+->]
  \item Queue: dequeue currently oldest (lowest time-stamp)
  \item Stack: pop currently newest (highest time-stamp)
  \end{itemize}
\item Priority queue generalizes this
\item Examples: highest priority, largest transaction, lowest price
\item Abstract from `item' and use only `key' (in applications:
use objects with fields {\tt item} and {\tt key}
and compare on {\tt key})
\end{itemize}
\end{frame}

\begin{frame}[fragile]
    \frametitle{Priority Queues}

\begin{itemize}[<+->]
\item Good info: \href{https://en.wikipedia.org/wiki/Priority_queue}%
{\color{red}{Wikipedia}}; API (the bare essentials):
\end{itemize}
{\tt\begin{tabular}{ll}
public class &MaxPQ<Key extends Comparable<Key>> \\
 \\
void       & insert(Key v)         // insert a key \\
%Key        & max()                      // return the largest key, if any\\
Key         &   delMax()                // delete a largest key, if any\\
boolean& isEmpty()\\
int          &size()
\end{tabular}}
\begin{itemize}[<+->]
\item In case of duplicate keys: `a' largest, not `the'
\item Typical application: the 1K smallest keys of 1G unsorted keys
\item Client:
\href{\git/programs/sorting/priorityQueues/BottomM.java}%
{\color{red}{BottomM.java}} (Q: why is the output slowing down?)
\end{itemize}
\end{frame}

\begin{frame}
    \frametitle{Example of left-complete binary tree}
\[
\xymatrix@R=.4cm@C=.4cm{
depth&&&&&&&&&&size\\ 
0&&&&&\bullet\ar[dll]\ar[drr] &&&&&1\\ 
1&&&\bullet \ar[dl] \ar[dr]&&&&\bullet \ar[dl] \ar[dr] &&&3\\
2&&\bullet \ar[dl]&&\bullet&&\bullet&&\bullet&&7\\
3&\bullet&&\ldots&&&&\ldots&&.& 7+1\\
\\
max:&|&\ar[l]&&&2^3&&&\ar[r]&|&\\
}
\] 
\end{frame}

\begin{frame}
    \frametitle{Binary Trees}

\begin{itemize}[<+->]

\item MNF130: Tree \emph{size} is number of nodes, 
\emph{depth} of a node is number of links to the root,
tree \emph{height} is maximum depth.

\item MNF130: In a \emph{binary} tree every node has at most two children.

\item MNF130: A binary tree is \emph{complete} if all levels are filled. So, a complete
binary tree of height $h$ has $2^{h{+}1} {-}1$ nodes.

\item INF102: A binary tree is (left-)\emph{complete} if all levels $<h$ are filled, 
level $h$ may be partly empty from the right (picture bb).
A (left-)complete binary tree of height $h$  has between
$2^{h}$ and $2^{h{+}1} {-}1$ nodes. 

\item A left-complete binary tree of $n$ nodes has  height $\lfloor\lg n\rfloor$
(from now on we leave out `left-' ).

\end{itemize}
\end{frame}

\begin{frame}
    \frametitle{Heap-ordered Binary Trees}
    
\begin{itemize}[<+->]

\item Naive implementations:
\begin{itemize}[<+->]
\item Unsorted (resizing) array: fast {\tt insert()}, linear {\tt delMax()}
\item Sorted (resizing) array: linear {\tt insert()}, fast {\tt delMax()}
\end{itemize}
\item Aim: operations in logarithmic time,  no extra space
\item A binary tree is \emph{heap-ordered} if the key in each node is $\geq$
the keys in its children (if any). Thus the root has a maximal key.
\item NB: a heap is NOT a search tree (different data invariants)!
\item Array representation of heap-ordered complete binary tree (bb)

\item Methods {\tt swim()} and {\tt sink()}: picture on bb, code below

\item Implementation:
\href{\git/programs/sorting/priorityQueues/ArrayListPQ.java}%
{\color{red}{ArrayListPQ.java}}
\end{itemize}     
\end{frame}

\begin{frame}
    \frametitle{Run-time and memory use of heaps, applications}

\begin{itemize}[<+->]
\item In a heap of $n$ elements (since height is $\leq \lfloor\lg n\rfloor$):
  \begin{itemize}[<+->]
  \item {\tt insert()} takes $\leq 1+\lfloor\lg n\rfloor$ compares
  and exchanges % $\leq \lfloor\lg n\rfloor$ 
  \item {\tt delMax()} takes $\leq 2\lfloor\lg n\rfloor$ compares
   and $\leq 1{+}\lfloor\lg n\rfloor$ exchanges
  \end{itemize}
\item Heap construction by {\tt insert()} can sometimes be improved
\item Example: maxheap from A B C D E F G H
\item Given an array of $n$ keys, {right-to-left} heap construction (bb) 
           takes $<2n$ compares and $<n$ exchanges
\item Applications: \href{\git/programs/sorting/priorityQueues/NaiveHeapSort.java}%
{\color{red}{heapsort}} and
\href{\git/programs/sorting/priorityQueues/MultiwayMerge.java}%
{\color{red}{merging sorted streams}}
\item Many variations with extended API (indexed priority queue)
\end{itemize}
\end{frame}

\section{Ch.2.5 Applications}

\begin{frame}
    \frametitle{Purpose of Sorting}

\begin{itemize}[<+->]
\item Sorting makes the following easier and more efficient:
  \begin{itemize}[<+->]
  \item Searching (binary search, example: {\tt ThreeSumOptimized}
  \item Searching and looking up, e.g., the pagenumber in an index
  \item Finding and removing duplicates
  \item Finding the median, quartiles etc.
  \end{itemize}
\item Our sorting algorithms are generic: {\tt sort(Comparable[] a)},
for any user-defined data type with a {\tt compareTo()} method
\item We do \emph{pointer sorting}, manipulating refs to objects. 
  \begin{itemize}[<+->]
  \item Pro: not moving full objects
  \item Cons: pointer dereferencing, no {\tt sort(int[] a)}
  \end{itemize}
\item More flexibility: pass a {\tt Comparator} object to {\tt sort()}
\end{itemize}     
\end{frame}

\begin{frame}[fragile]
    \frametitle{Comparator object}

\begin{itemize}[<+->]
\item API: {\tt void sort(Object[] a, Comparator c)}
\item Call, e.g.: {\tt sort(a, new Transaction.WhenOrder())}
\item Call, e.g.: {\tt sort(a, new Transaction.SizeOrder())}
\item Obs: {\tt import java.util.Comparator}
\item Obs: {\tt less(Object o1, Object o2, Comparator c)}
\item Priority queues also with {\tt Comparator}
\end{itemize}
\begin{verbatim}
public class Transaction {
 ...
 public static class MyOrder {
 implements Comparator<Transaction>
  public int compare(Transaction t, Transaction v){...}
 } // End of Myorder
...// similarly: WhenOrder, SizeOrder
} // End of Transaction
\end{verbatim}    
\end{frame}

\begin{frame}
    \frametitle{Applications of Sorting}

\begin{itemize}[<+->]
\item Consider sorting first to make other problems easier
\item Commercial computing (sort on price, departure time, ...)
\item Search for information: web-indexing, search engines
\item Job scheduling heuristic: longest processing time first 
%\item Combinatorial search in AI
\item To come: Prim's, Dijkstra's  and Kruskal's algorithms
\item Huffman compression: a lossless compression based on using the
shortest codes for the symbols that occur oftest. Frequency counter:
next chapter!
\item Cryptology and genomics (e.g., longest repeated substring)
\end{itemize}     
\end{frame}

\section{Ch.3.1 Symbol Tables}

\begin{frame}
    \frametitle{Symbol Tables}

\begin{itemize}[<+->]
\item Symbol table associates \emph{keys} with \emph{values: key-value pairs}
\item Examples: keyword-list of page nrs, ID number-personal data, word-frequency
\item Important operations:
  \begin{itemize}[<+->]
  \item Insert a key-value pair in the symbol table: {\tt void put(k,v)}
  \item Search the value for a given key (if any): {\tt Value get(k)}
  \end{itemize}
\item Important conventions:
  \begin{itemize}[<+->]
  \item Inserting key-value for existing key: overwriting the value
  \item No duplicate keys, no null keys
  \item Value null: no value for this key
  \item Lazy deletion: insert key-null; Eager: really delete key-value
  \end{itemize} 
\item\href{http://algs4.cs.princeton.edu/code/javadoc/edu/princeton/cs/algs4/SequentialSearchST.html}%
{\color{red}{API}} of unordered symbol table
\item Aim: all operations in time $\sim c \lg n$ with constant $c$ small
\end{itemize}     
\end{frame}

\begin{frame}
    \frametitle{ST Basics}

\begin{itemize}[<+->]
\item Archetypical ST-client: frequency counter, \href{\git/programs/searching/elementarySymbolTables/ArrayListST.java}%
{\color{red}{ArrayListST.main}}
\item Cost model: number of compares
\item Naive ST: unordered linked list (or ArrayList), linear search %(INF101, Ch.9)
  \begin{itemize}[<+->]
  \item Search miss: $\sim n$ compares
  \item Search hit: between 1 and $\sim n$ compares
  \item Random search hit: $(1+\cdots+n)/n \sim n/2$ compares
  \item Inserting $n$ distinct keys: $(1+\cdots+(n-1)) \sim n^2/2$ compares
  \end{itemize}
\item \url{algs4-data/leipzig1M.txt}: 21M words, 500K distinct
\item Naive ST impracticable for genomics, internet
\item Scale: G-T keys, M-G distinct (Kilo,Mega,Giga,Tera)
\item Better for unordered ST: hashing (in Ch. 3.4)
\end{itemize}     
\end{frame}

\begin{frame}
    \frametitle{Ordered Symbol Table}

\begin{itemize}[<+->]
\item Ordered ST: keys are ordered, f.e., in an ArrayList
\item\href{http://algs4.cs.princeton.edu/code/javadoc/edu/princeton/cs/algs4/BinarySearchST.html}%
{\color{red}{API}} of ordered symbol table
\item Binary search: {\tt get(Key k)} takes $\sim \lg n$ comparisons
\item What about {\tt put(Key k, Value v)}? See
\href{http://docs.oracle.com/javase/7/docs/api/java/util/ArrayList.html}%
{\color{red}{ArrayList}}
\item Pitfall: {\tt add(int i, E e)} is linear, not amortized $O(1)$!
\item Consequence: {\tt put(Key k, Value v)} and {\tt del(Key k)} \emph{linear}
\item Implementation with binary search in
\href{\git/programs/searching/elementarySymbolTables/ArrayListST.java}%
{\color{red}{ArrayListST.java}}
\item Trace of inserts on bb: S E A R C H E X A M P L E
\item Experiments with {\tt tinyTale.txt}, {\tt tale.txt}, ... 
\end{itemize}     
\end{frame}

\section{Ch.3.2 Binary Search Trees}

\begin{frame}
    \frametitle{Binary Search Trees}

\begin{itemize}[<+->]
\item Aim: {\tt get,put,del} in logarithmic time, ST in linear space
\item Binary \emph{search} tree: for every node, all keys to the left of this node
are smaller, and all keys to the right are larger
\item Search time: lenght of the path to the node where the key `should' be
\item Balanced binary tree with $n$ keys has $\lg n$ height
\item Unbalanced binary trees can have height $n$ (max depth)
\item Search hits in a binary search tree, built without rebalancing, 
of $n$ random keys take on average $\sim 2\ln n$ compares
\item \href{\git/programs/searching/elementarySymbolTables/UBST.java}%
{\color{red}{UBST.java}}: {\tt put(), get(), size(), isEmpty()}
\item Trace of inserts on bb: S E A R C H E X A M P L E
\end{itemize}     
\end{frame}

\begin{frame}[fragile]
    \frametitle{Binary Search Trees ({\tt min(), delMin(), delete()})}

\begin{itemize}[<+->]
\item Interrelated, increasing difficulty: {\tt min(Node x),\\
deleteMin(Node x), delete(Node x, Key k)}
\item Node of minimum key: not {\tt null}, and has left child {\tt null}, and is root or 
left child of parent (picture on bb)
\begin{verbatim}
public Node min(Node x){// precondition x!=null
 while (x.left!=null) x = x.left; // inv x!=null
 return x; 
} // cf. tail recursive min() in Alg. 3.3
\end{verbatim}
\item Delete minimum key, two cases:\\ (1) both children {\tt null}; (2) left child {\tt null} 
\item Delete is really difficult: \href{http://algs4.cs.princeton.edu/code/edu/princeton/cs/algs4/BST.java.html}%
{\color{red}{BST.java}},
cf.\ \href{\git/programs/searching/elementarySymbolTables/ArrayListST.java}%
{\color{red}{ArrayListST.java}}
\item Don't forget: update {\tt x.N} along the path to the root!
\end{itemize}     
\end{frame}



\begin{frame}[fragile]
    \frametitle{Delete from search tree, example:
$\xymatrix@R=.1cm@C=.1cm{&&&2\ar@{->}[dll]\ar@{->}[rrd]&&\\
&1\ar@{-|}[dl]\ar@{-|}[dr]&&&&3\ar@{-|}[dl]\ar@{-|}[dr]&\\&&&&&&}$              }
\begin{verbatim}
root=delete(root,3)                     (1st example)
| x=root; x.right=delete(x.right,3) 
  | x'=x.right; return x'.left;         (x.right=null)
| update x.size;                        (x.size=2)
| return x;                             (root=x)
\end{verbatim}

\begin{verbatim}
root=delete(root,2)                     (2nd example)
| x=root; t=x; x=min(t.right);          (x=t.right)
| x.right=deleteMin(t.right);           (x.right=null)
| x.left=t.left;
| update x.size;                        (x.size=2)
| return x;                             (root=x)
\end{verbatim}

\end{frame}
%\begin{itemize}[<+->]
%\item 
%\item {\tt get,put,del} in logarithmic time, ST in linear space
%\end{itemize}     

\section{Ch.3.3 Balanced Search Trees}

\begin{frame}
    \frametitle{Balanced Search Trees: keep paths short!}

\begin{itemize}[<+->]
\item NB tree balancing not as easy as in UF and Heap (4hrs!)
\item A \href{https://en.wikipedia.org/wiki/2-3-tree}{\color{red}{2-3 search tree}} 
consists of 2-nodes and 3-nodes: 
  \begin{itemize}[<+->]
  \item Each 2-node has two children and a key $k$ such that all keys
  in the left subtree are $<k$, and all keys in the right subtree $>k$
  \item Each 3-node has three children and two keys $k_1 < k_2$ such that all keys
  in the left subtree are $<k_1$, all keys in the middle subtree
  $>k_1$ and $<k_2$, and all keys in the right subtree $>k_2$
  \end{itemize}
\item Examples and pictures on bb
\item \emph{Perfect}  2-3 search tree: paths from root to leaves equally long
\item Search: compare key with key(s) in node, if equal return corresponding value,
  else search in one of left, middle, right subtree where the key should be 
  (if it occurs at all)
\item Insert should keep tree perfect, rough idea:  
 \begin{itemize}[<+->]
 \item into a 2-leaf: make it into a 3-leaf (easy)
 \item any other case: (temporary) 4-nodes (see next, difficult)
 \end{itemize}     
\end{itemize}     
\end{frame}

\begin{frame}
    \frametitle{Insert in Perfect 2-3 Search Trees}

\begin{itemize}[<+->]
\item Terminology: a \emph{leaf} is a node all whose children are {\tt null}
\item Data invariant 1: tree is 2-3 search tree
\item Data invariant 2: all paths from root to leaves equally long
 \begin{itemize}[<+->]
 \item Insert into a 2-leaf 
 $\xymatrix@R=.1cm@C=.1cm{L \ar@{=}[d]&:\\&\\}$
 either
 $\xymatrix@R=.1cm@C=.1cm{&A~L\ar@{-}[dl]\ar@{-}[d]\ar@{-}[rd]&\\&&\\}$
 or
 $\xymatrix@R=.1cm@C=.1cm{&L~Z\ar@{-}[dl]\ar@{-}[d]\ar@{-}[rd]&\\&&\\}$
 \item  into a 3-leaf whose parent is a 2-node: with new key $Z$ (e.g.)
 $\xymatrix@R=.1cm@C=.1cm{
 &\ar@{-}[dl]E\ar@{-}[rd]&&\Rightarrow\\
 leaf\ar@{=}[d]&&L~Q\ar@{-}[dl]\ar@{-}[d]\ar@{-}[rd]&\\
 &&&\\}$
 $\xymatrix@R=.1cm@C=.1cm{
 &\ar@{-}[dl]E\ar@{-}[rd]&&\Rightarrow\\
 leaf\ar@{=}[d]&&L~Q~Z\ar@{-}[dl]\ar@{-}[d]\ar@{-}[rd]&\\
 &&&\\}$
 $\xymatrix@R=.1cm@C=.1cm{
 &\ar@{-}[dl]E~~Q\ar@{-}[d]\ar@{-}[rd]&\\
 leaf\ar@{=}[d]&L\ar@{=}[d]&Z~\ar@{=}[d]\\
 &&\\}$
  \item  into a 3-leaf whose parent is a 3-node: with new key $Z$ (e.g.)
 $\xymatrix@R=.1cm@C=.1cm{
 &&\ar@{-}[dll]C~~M\ar@{-}[d]\ar@{-}[rrd]&&&\Rightarrow\\
 leaf_l\ar@{=}[d]&&leaf_m\ar@{=}[d]&&\ar@{-}[ld]XYZ\ar@{=}[d]\ar@{-}[rd]&\\
 &&&&&&\\}$
 $\xymatrix@R=.1cm@C=.1cm{
 &&\ar@{-}[dll]C~~~M\ar@{-}[d]&Y\ar@{-}[d]\ar@{-}[rd]&&\\
 leaf_l\ar@{=}[d]&&leaf_m\ar@{=}[d]&\ar@{=}[d]X&Z\ar@{=}[d]&\\
 &&&&&&\\}$
 \item  into a 3-node whose parent is a 3-node: move up middle key!
% \item If all 3-nodes: split the root
 \end{itemize}     
\end{itemize}     
\end{frame}

\begin{frame}
    \frametitle{Insert (ctnd)}

\begin{itemize}[<+->]
\item Data invariant 1: tree is 2-3 search tree
\item Data invariant 2: all paths from root to leaves equally long
\item Insert works up from the  leaf where the key `should' be
\begin{itemize}[<+->]
 \item if 2-node on path to root: make it into a 3-node;  
 there are two cases, left and right, here is a picture of the latter
 $\xymatrix@R=.1cm@C=.1cm{
 &&\ar@{-}[dll]E\ar@{-}[rrd]&&&&&\Rightarrow\\
 t\ar@{.}[d]&&&L\ar@{-}[dl]&Q\ar@{-}[dl]\ar@{-}[rd]&Z\ar@{-}[rd]&\\
 \ldots&&t_1&t_2&&t_3&t_4\\}$
 $\xymatrix@R=.1cm@C=.1cm{
 &&\ar@{-}[dll]E\ar@{-}[rd]&Q\ar@{-}[dr]&&\\
 t\ar@{.}[d]&&&L\ar@{-}[dl]\ar@{-}[d]&Z\ar@{-}[d]\ar@{-}[rd]&\\
 \ldots&&t_1&t_2&t_3&t_4\\ \\}$
 
 \item otherwise, work upwards and, finally, split the root: 
 $\xymatrix@R=.1cm@C=.1cm{
&&&&&\Rightarrow\\
 &L\ar@{-}[dl]&Q\ar@{-}[dl]\ar@{-}[rd]&Z\ar@{-}[rd]&\\
 t_1&t_2&&t_3&t_4\\}$
 $\xymatrix@R=.1cm@C=.1cm{
 &&\ar@{-}[dl]Q\ar@{-}[rd]&&\\
 &\ar@{-}[d]L\ar@{-}[dl]&&Z\ar@{-}[d]\ar@{-}[rd]&&\\
 t_1&t_2&&t_3&t_4&\\ \\}$

\item working upwards, there are three cases: left, middle, and right
 \end{itemize}  
 
 \end{itemize}     
\end{frame}

\begin{frame}
    \frametitle{Insert, summary and examples}

\begin{itemize}[<+->]
\item There exists a perfect 2-3 tree for any sequence of input keys 
\item Six operations for eliminating 4-nodes:
  \begin{itemize}[<+->]
  \item if parent is 2-node: move middle key up (left and right case)
  \item if parent is 3-node: move middle key up (left, middle,right)
  \item if root: split root
  \end{itemize}
\item Search and insert visit at most $\lfloor\lg n\rfloor$ nodes
\item Proof: maximal path length is $\geq \lfloor\log_3 n\rfloor$ 
and $\leq \lfloor\log_2 n\rfloor$
\item Trace of inserts on bb: S E A R C H (E) X (A) M P L (E)
\item Trace of inserts on bb: A C E H L M P  R S X (keep balance!)
\end{itemize}     
\end{frame}

\begin{frame}
    \frametitle{Red-black trees}

\begin{itemize}[<+->]
\item Red-black trees implement 2-3 trees
\item Idea: one 3-node = two 2-nodes + extra info
\item Extra info coded in color, picture:
\[\xymatrix@R=.5cm@C=.1cm{
&L\ar@{-}[dl]\ar@{-}[d]~R\ar@{-}[rd]&&implemented~as
&&L\ar@{>}[dl]\ar@{>}[rd]&&\ar@{>}@[red][ll]R\ar@{>}[rd]&\\
 t_l&t_m&t_r
&& t_l&&t_m&&t_r \\}
\]
\item A \emph{red-black tree} is a binary search tree with red and black
links such that: 
  \begin{itemize}[<+->]
  \item Only left links can be red (but need not be)
  \item Never $\xymatrix@R=.1cm@C=.5cm{&\ar@{>}@[red][l]&\ar@{>}@[red][l]}$
  \item Perfect black balance (all paths from root to leaves same number of black
  links; this number is called the \emph{black height})
  \end{itemize}
\item Equivalent: red-black tree and perfect 2-3 search tree
 \end{itemize}     
\end{frame}

\begin{frame}[fragile]
    \frametitle{Red-black trees (ctnd)}

\begin{itemize}[<+->]
\item Color is attribute of \emph{incoming} link (why?)
\begin{verbatim}
private class Node {
  Key key;
  Value value;
  Node left, right;
  boolean color; // true for red, false for black
  int N; // number of keys in subtree below this node
}
private boolean isRed(Node n) {
 if (n==null) {return false;} else {return x.color}
\end{verbatim}
\end{itemize}     
\end{frame}

\begin{frame}[fragile]
    \frametitle{Rotating and Color Flipping}

\begin{itemize}[<+->]
\item Aim: restoring the data invariants of red-black search trees
  \begin{enumerate}[<+->]
  \item Only left links can be red, but never two successive
  \item Search tree invariant
  \item Perfect black balance
  \end{enumerate}
\item Repairments use \emph{rotations} and \emph{color flips}, examples (leaves):
  \begin{itemize}[<+->]
  \item inserting $Z$ in
  $\xymatrix@R=.1cm@C=.5cm{L&\ar@{>}@[red][l]R}$:
  $\xymatrix@R=.1cm@C=.5cm{L&\ar@{>}@[red][l]R\ar@{>}@[red][r]&Z}$,
  violation (why?)\\
 repairment:  color flip
  $\xymatrix@R=.1cm@C=.5cm{L&\ar@{>}[l]R\ar@{>}[r]&Z}$ ($R$ must up)

  \item inserting $A$ in
  $\xymatrix@R=.1cm@C=.5cm{L&\ar@{>}@[red][l]R}$:
  $\xymatrix@R=.1cm@C=.5cm{A&\ar@{>}@[red][l]L&\ar@{>}@[red][l]R}$,
  violation (why?)\\
  repairment:
  rotation right + color flip
  $\xymatrix@R=.1cm@C=.5cm{A&\ar@{>}[l]L\ar@{>}[r]&R}$

  \item inserting $M$ in
  $\xymatrix@R=.1cm@C=.5cm{L&&\ar@{>}@[red][ll]R}$:
  $\xymatrix@R=.1cm@C=.5cm{L\ar@{>}@[red][rd]&&\ar@{>}@[red][ll]R\\&M}$,
  violation\\
  repairment:
  rotation left into
  $\xymatrix@R=.1cm@C=.5cm{L&\ar@{>}@[red][l]M&\ar@{>}@[red][l]R}$,
  then as above
  \end{itemize}
  \item Rotations change the root of the subtree, preserving invariants
\end{itemize}     
\end{frame}

\begin{frame}[fragile]
    \frametitle{Left Rotation}
Call: {\tt l = rotateLeft(l); }

%\begin{itemize}[<+->]
\[\xymatrix@R=.5cm@C=.1cm{
&&\ell \ar@{>}[d]&&&&&& \ell \ar@{>}[rrd]\\
&&L\ar@{>}[dl]\ar@{>}@[red][rr]&&\ar@{>}[ld]R\ar@{>}[rd]&&\Rightarrow
&&L\ar@{>}[dl]\ar@{>}[rd]&&\ar@{>}@[red][ll]R\ar@{>}[rd]&\\
& t_l&&t_m&&t_r
&& t_l&&t_m&&t_r \\}
\]
\begin{verbatim}
private Node rotateLeft(Node l){
  Node r = l.right; l.right = r.left; r.left = l;
  r.color = l.color; l.color = true // == RED 
  r.N = l.N; l.N -= 1+size(r.right); // Why?
  return r;
}
\end{verbatim}
%\end{itemize}     
\end{frame}

\begin{frame}[fragile]
    \frametitle{Right Rotation and Color Flip}
Typically in the following situation (e.g., after insert(L) in a 3-leaf):
\[\xymatrix@R=.5cm@C=.1cm{
&&r \ar@{>}[rrd]&&&\mbox{R.R.}&&& r \ar@{>}[d]&&&\mbox{C.F.}&&& r \ar@{>}@[red][d]\\
\ar@{.}[d]L&&\ar@{>}@[red][ll]\ar@{>}[d]M&&\ar@{>}@[red][ll]R\ar@{>}[d]&\Rightarrow&
L\ar@{.}[d]&&\ar@{>}@[red][ll]M\ar@{>}@[red][rr]&&\ar@{>}[lld]R\ar@{>}[d]&\Rightarrow&
L\ar@{.}[d]&&\ar@{>}[ll]M\ar@{>}[rr]&&\ar@{>}[lld]R\ar@{>}[d]\\
\ldots&&t_m&&t_r&&
\ldots&&t_m&&t_r &&
\ldots&&t_m&&t_r
}
\]
\begin{itemize}[<+->]
\item Code of {\tt rotateRight()}like that of {\tt rotateLeft()}
\item NB1: operations are local (here only {\tt r, M , R})
\item NB2: operations preserve data invariants
\item NB3: root is a special case (always black)
\item Deletions: complicated, but doable (Exc.\ 3.3.39--41)
\end{itemize}     
\end{frame}

\begin{frame}
    \frametitle{Run-time and memory use of Red-Black BSTs}

\begin{itemize}[<+->]
\item The height of a red-black BST with $n$ nodes is $\leq 2\lg n$\\
Proof: the worst-case is one 3-node path and the rest 2-nodes
\item The average length of path  (any color) from the root to a {node}
in a red-black BST with $n$ nodes is $\lg n$ (`empirical fact')
\item In a red-black BST, search, insert, ..., and delete, take logarithmic time
in the worst-case. Proof: a constant amount of work is done per visited node
(book Prop.\ I, p. 447).
\item For red-black BSTs, logarithmic time is guaranteed!
\end{itemize}
\end{frame}

\section{Ch.3.4 Hash Tables}

\begin{frame}
    \frametitle{Hashing}

\begin{itemize}[<+->]
\item Idea: if keys in {\tt [0..99]} an array is the perfect symbol table
\item In fact:
\href{\git/programs/sorting/elementarySorts/CountSort99.java}%
{\color{red}{CountSort99.java}} counts frequencies like an ST client
\item A \emph{hash function} maps keys to array indices
\item Injectivity of the hash function is not guaranteed
\item \emph{Hash collision}: different keys are mapped to the same index
\item In such a case we need \emph{collision resolution}
%\item Hashing  = hash function + collision resolution
\item Symbol tables: hashing is fast, but unordered (no {\tt max,min})
\item Aim: ST operations in amortized $O(1)$ time, extra space OK
\end{itemize}
\end{frame}

\begin{frame}
    \frametitle{Space-Time Trade-Off}

\begin{itemize}[<+->]
\item Hashing is an example of a \emph{space-time trade-off}
\item Time: computation time required %(may use little space)
\item Space: memory space used %(which also takes time)
\item Unlimited space: (1) use key as index (e.g., the bits)
\item Unlimited time: (2) use linked list and linear search 
\item Hashing strikes a balance using (1) with some array of reasonable size,
and (2) in case of collisions
\item The balance between (1) and (2) can easily be tuned
\end{itemize}
\end{frame}

\begin{frame}[fragile]
    \frametitle{Hash functions}

\begin{itemize}[<+->]
\item Ideal (uniform hashing assumption, UHA): uniform and independent
distribution of keys over integers from 0 to $M-1$
\item Examples of \href{https://en.wikipedia.org/wiki/Java_hashCode()}%
{\color{red}{hash functions in Java}}
\item Horner: $a_0 + x ( a_1 + x ( a_2 + \cdots))= a_0 + a_1 x + a_2 x^2 + \cdots$ 
\item Modular hashing ($M$ prime), reasonably $\approx$ UHA:
\begin{verbatim}
private int hash(Key k){
 return (key.hashCode() & 0x7fffffff) % M;}
\end{verbatim}
\item Q: Why crazy {\tt \& 0x7fffffff}  ???
\item A: In Java, e.g., {\tt (-5 \% 3) == -2}) and not 1 
\item Q: Why $M$ prime? 
\item A: E.g., $M=32$ takes only into account the last five bits
\end{itemize}
\end{frame}

\begin{frame}
    \frametitle{Collision Resolution}

\begin{itemize}[<+->]
\item Two main methods of collision resolution:
  \begin{enumerate}[<+->]
  \item Hashing with separate chaining (picture on bb)
  \item Hashing with linear probing (picture on bb)
  \end{enumerate}
\item Separate chaining: symbol table is an array of linked lists,
  linear search. If array has length $M$, 
  then the linked lists have average length $N/M$ with $N$ keys.
\item Linear probing: symbol table is an array of length $M\geq N$.
  Colliding keys are put at the first empty position. Linear search from
  the position where the key `should have been'. Empty position: not found.
  Deletion tricky: reinsert all keys to the right of the deleted key, until the first
  empty position (picture on bb). 
  Works better with $M >> N$.
\end{itemize}
\end{frame}

\begin{frame}[fragile]
    \frametitle{Symbol Table with Hashing}

\begin{itemize}[<+->]

\item Implementation: \href{\git/programs/searching/hashTable/ArrayListHashST.java}%
{\color{red}{\tt ArrayListHashST.java}}
\item $M=1$: measure overhead wrt.\ {\tt ArrayListST}
\item Tests with various values of  $M$: 31, 997, 65521
\item NB1: {\tt ArrayListST, UBST} are sorted (!)
\item NB2: \emph{construction} versus \emph{use} of ST (hashing better for \emph{use})
\item Hashing can be combined with any other ST-implementation
\item UHA metaphor: for every key one throws a dice \emph{once}, and remembers
the value as the hash code of the key 
\end{itemize}
\end{frame}

\begin{frame}[fragile]
    \frametitle{Quantitative analysis}

\begin{itemize}[<+->]
\item Throwing a dice 10 times, what is the probability of 3 fives?
\item Under UHA, with $N$ distinct keys, the probability that exactly $k$ keys collide at some given hash value is
\[ \binom{N}{k} (\frac{1}{M})^k  (\frac{M-1}{M})^{N-k},\text{ where e.g. } \binom{100}{10}\approx 1.7 E 13\]
\item This is a small number for, say, $N=M=100$ and $k=10$
\item For linear probing one typically takes $M=2N$
\item For separate chaining one keeps $N/8 \leq M \leq N/2$ (resizing $M$)
\item Under UHA: search, insert, delete take amortized $O(1)$ time
\item Space used can be upto $100N$ byte (objects, pointers); 
this on top of the space used by $N$ key-value pairs
\end{itemize}
\end{frame}


\section{Ch.3.5 Applications of Searching}

\begin{frame}[fragile]
    \frametitle{Applications of Searching}

\begin{itemize}[<+->]
\item Synonyms: 
\href{https://en.wikipedia.org/wiki/Associative_array}{\color{red}{associative array}}, 
map, symbol table, or dictionary
\item Origin of
\href{https://en.wikipedia.org/wiki/Symbol_table}{\color{red}{symbol table}}: 
compilers and interpreters
\item Web-indexing, \href{https://en.wikipedia.org/wiki/Search_engine_indexing#Inverted_indices}{\color{red}{search engines}}
\item Sparse matrices (many 0's): \href{https://en.wikipedia.org/wiki/Sparse_matrix#Dictionary_of_keys_.28DOK.29}%
{\color{red}{dictionary}} 
  \begin{enumerate}[<+->]
  \item keys: (row, column)-pairs
  \item values: matrix entries
  \end{enumerate}
\item Set API (no values, only keys, for deduplication, filtering):
\end{itemize}
{\tt\begin{tabular}{ll}
public class &SET<Key> \\
 \end{tabular}\\
\begin{tabular}{rl}
\{  void       & add(Key k); \\
   void       & delete(Key k); \\
   boolean    & contains(Key k); \\
   boolean    & isEmpty();\\
   int        & size();              \}
\end{tabular}
}
\end{frame}

\begin{frame}[fragile]
    \frametitle{Indexes and Reverse Indexes}

\begin{itemize}[<+->]
\item Index (key, value)
\item Reverse index (value, key(s))
\item Phone book (name+address, phone number(s))
\item Reverse phone book (phone number, name+address)
\item Dictionary (word, meaning or translation)
\item Account information (client ID, account information)
\item Genomics (protein, sequences of ACTG triplets)
%\item Experimental data of various kinds
\item File systems (file name, location on disk/ file attributes)
\item Internet domain name system (domain name, IP address)
\end{itemize}
\end{frame}


\begin{frame}[fragile]
    \frametitle{Balanced Search Tree or Hash Table?}

\begin{itemize}[<+->]
\item Q: Which symbol table to use?
\item A: The basic choice between BST and HT depends on ...
  \begin{enumerate}[<+->]
  \item Ordering of keys essential: BST
  \item Availability of good hash function (good = fast + UHA)
  \item Ordering of keys expensive (long strings): HT (or: Ch.5)
  \item Ordering of keys possible, but not essential: HT + BST
  \item Space considerations (ArrayListST uses the least extra space)
  \item Number of distinct keys and the space each key takes
  \item Distribution of insert/delete/search operations
  \end{enumerate}
\end{itemize}
\end{frame}

\section{Overview Chapter 1--3}

\begin{frame}[fragile]
    \frametitle{Overview Chapter 1--3}

\begin{enumerate}[<+->]
\item[] Chapter 1
 \begin{itemize}[<+->]
 \item Stack and Queue, ThreeSum, Union-Find
 \item Theory: $\sim$ and $O$
 \item Experiments: loglog-plots, randomization 
 \end{itemize}
\item[] Chapter 2: Sorting
 \begin{itemize}[<+->]
 \item Selection-, Insertion-, Shell-, Merge-, QuickSort
 \item Priority Queue, Binary Heap, HeapSort
 \item CountSort
 \end{itemize}
 \item[] Chapter 3
 \begin{itemize}[<+->]
 \item Symbol Table
 \item Binary Search Tree, Perfect 2-3 Tree, Red-Black Tree
 \item Hashing: hash function and collision resolution
 \end{itemize}
\end{enumerate}
\end{frame}

\begin{frame}[fragile]
    \frametitle{Odds and Ends Chapter 1--3}

\begin{itemize}[<+->]
\item Path-compression in UF
(\hyperlink{pathcompression}{\color{red}{\ref{pathcompression}}})
%\item Find out if {\tt add(int i, E e)} in ArrayList is amortized $O(1)$
\item Randomizing the order of the tests
\item Compare-based sorting requires $N \lg N$ comparisons
(\hyperlink{comparebasedsorting}{\color{red}{\ref{comparebasedsorting}}})
%\item Discuss primitive types (objects are costly)
%\item \href{https://en.wikipedia.org/wiki/Distributed_hash_table}{\color{red}{Distributed Hash Table}}
\item Double hashing: linear probing $h_1(k), h_1(k)+h_2(k), h_1(k)+2h_2(k), \ldots$
\item Indexed Priority Queues
(\hyperlink{indexedprioqueues}{\color{red}{\ref{indexedprioqueues}}})
\end{itemize}
\end{frame}

\begin{frame}[fragile]
    \frametitle{Randomizing the order of the tests}
\begin{verbatim}
for (int i=0; i<N; i++) {
    a1[i] = StdRandom.uniform();
    a2[i] = a1[i]; }
if (StdRandom.uniform(2) == 0)
   {total1 += time(alg1, a1); // summing runtime of alg1
    total2 += time(alg2, a2); // summing runtime of alg2
    }
else 
   {total2 += time(alg2, a2); // summing runtime of alg2
    total1 += time(alg1, a1); // summing runtime of alg1
    }
\end{verbatim}
\end{frame}

\begin{frame}[fragile]
    \frametitle{Path compression in UF}\label{pathcompression}
\begin{verbatim}
// Finding the "identifier" of the component of p in id:
public int find(int p) {
  while (p!=id[p]) { p=id[p]; }
  return p;
}  
// now with path compression:

public int find(int p) {
  int q=p; // remember the starting point
  while (p!=id[p]) { p=id[p]; }
  // postcondition: p==id[p]==identifier of q
  while (id[q]!=p) { int aux=id[q]; id[q]=p; q=aux; }
  return p;
} // Example: int[] id={1,2,3,3}; find(0);
\end{verbatim}
\end{frame}

\begin{frame}[fragile]
    \frametitle{Compare-based sorting: worst-case $\geq N\lg N$}
\label{comparebasedsorting}
\begin{itemize}[<+->]
\item Every compare-based sorting algorithm for $N$ distinct keys
in an array $a$ leads to a \emph{binary compare tree} with 
 \begin{itemize}[<+->]
 \item nodes $(i{:}j)$ representing tests $a[i]<a[j]$
 \item left subtree: $a[i]<a[j]$; right subtree: $a[i]>a[j]$
 \item leaves: sorted permutations of the array
 \end{itemize}
\item Example with array of length 3 on bb
\item Every permutation should occur at least once in a leaf!
\item Binary tree of height $h$ has at most $2^h$ leaves
\item Length of path to leaf = number of comparisons
\item Now $h\geq \lg N! \sim N \lg N$ by Stirling from this formula:
\end{itemize}
$$ N! = \text{number of permutations}  
     \leq \text{number of leaves} \leq 2^\text{height of tree} $$

\end{frame}

\begin{frame}
    \frametitle{Indexed Priority Queues (missing inverse {\tt qp} of {\tt pq})}
\label{indexedprioqueues}
\begin{itemize}[<+->]
\item IPQ ${\approx}$ array with direct access to minimum (maximum)
\item API: void insert(int i, Key k); void del(int i); int minKey(); Key keyOf(int i);...
Example of implementation:
\end{itemize}
$\xymatrix@R=.5cm@C=.1cm{
&&\text{heap}\\
&&A \ar@{>}[ld]\ar@{>}[rd]&&\\
&C\ar@{>}[ld]\ar@{>}[rd]&&F\\
D&&E&}
$
\begin{tabular}{c}
{\tt
\begin{tabular}{|r|c|c|c|c|c|c|c|}
\hline
index & 0&1&2&3&4&5&6 \\\hline
pq    & 1&0&2&4&3&0&0 \\\hline
keys  & C&A&F&E&D&-&- \\\hline
%qp    & 1&0&2&4&3&-1&-1\\\hline
\end{tabular}
}\\
\\
Do: insert(6,G), insert(5,B), insert(1,Z)\\
NB1: {\tt keys[pq[i]]} on position {\tt i} in heap\\
NB2: inverse index {\tt qp} is needed 
\end{tabular}
\end{frame}

\begin{frame}
    \frametitle{Indexed Priority Queues}
\label{indexedprioqueues}
\begin{itemize}[<+->]
\item IPQ ${\approx}$ array with direct access to minimum (maximum)
\item API: void insert(int i, Key k); void del(int i); int minKey(); Key keyOf(int i);...
Example of implementation:
\end{itemize}
$\xymatrix@R=.5cm@C=.1cm{
&&\text{heap}\\
&&A \ar@{>}[ld]\ar@{>}[rd]&&\\
&C\ar@{>}[ld]\ar@{>}[rd]&&F\\
D&&E&}
$
\begin{tabular}{c}
{\tt
\begin{tabular}{|r|c|c|c|c|c|c|c|}
\hline
index & 0&1&2&3&4&5&6 \\\hline
pq    & 1&0&2&4&3&0&0 \\\hline
keys  & C&A&F&E&D&-&- \\\hline
qp    & 1&0&2&4&3&-1&-1\\\hline
\end{tabular}
}\\
\\
Do: insert(6,G), insert(5,B)\\
NB  {\tt qp} is needed to find \\
the index of {\tt key[i]} in {\tt pq}\\
e.g., for insert(1,Z) (then: sink!)
\end{tabular}
\end{frame}


\begin{frame}
    \frametitle{Indexed Priority Queues (ctnd)}
After insert(6,G):\quad\quad
{\tt
\begin{tabular}{|r|c|c|c|c|c|c|c|}
\hline
index & 0&1&2&3&4&\color{red}{5}&6 \\\hline
pq    & 1&0&2&4&3&\color{red}{6}&0 \\\hline
keys  & C&A&F&E&D&-&\color{red}{G} \\\hline
qp    & 1&0&2&4&3&-1&\color{red}{5}\\\hline
\end{tabular}
}\\
Step 1 insert(5,B):\quad\quad
{\tt
\begin{tabular}{|r|c|c|c|c|c|c|c|}
\hline
index & 0&1&2&3&4&5&\color{red}{6} \\\hline
pq    & 1&0&2&4&3&6&\color{red}{5} \\\hline
keys  & C&A&F&E&D&\color{red}{B}&G \\\hline
qp    & 1&0&2&4&3&\color{red}{6}&5\\\hline
\end{tabular}
}\\
\begin{tabular}{ll}
Step 2 &(swaps)\\
pq[2]&pq[6]\\
qp[pq[2]] & qp[pq[6]]
\end{tabular}
{\tt
\begin{tabular}{|r|c|c|c|c|c|c|c|}
\hline
index & 0&1&\color{red}{2}&3&4&5&\color{red}{6} \\\hline
pq    & 1&0&\color{red}{5}&4&3&6&\color{red}{2} \\\hline
keys  & C&A&\color{red}{F}&E&D&\color{red}{B}&G \\\hline
qp    & 1&0&\color{red}{6}&4&3&\color{red}{2}&5\\\hline
\end{tabular}
}\\
%NB  {\tt qp} is needed to find the index of {\tt key[i]} in {\tt pq}!
\end{frame}


\section{Ch.4.1 Undirected Graphs}

\begin{frame}
    \frametitle{Graph classes}

( MNF130: useful review of graph theory)

\begin{enumerate}[<+->]\label{graph_classes}
\item Undirected graphs: a set of \emph{vertices} (or \emph{nodes}) $V$
and a set of \emph{edges} $E$ connecting the nodes
%$E \subseteq \set{\set{v,v'}\mid v\in V, v'\in V, v\neq v'}$
\item Directed graphs (\emph{digraphs}): 
a set of nodes $V$ and a set $E$ of directed edges (or \emph{arrows})
pointing from one node to another
\item \emph{Edge-weighted graphs}: undirected graphs in 
which every edge comes with a number called the \emph{weight} of the edge
\item \emph{Edge-weighted digraphs}: digraphs  in which every arrow has a weight,
like the edges of an edge-weighted graph
\end{enumerate}
\end{frame}

\begin{frame}
    \frametitle{Examples}

\begin{enumerate}[<+->]
\item \href{http://internationalstudentblog.b.uib.no/files/2015/06/Linjekart_sentrum__3962034a.png}%
{\color{red}{Map}} (discuss: nodes, un/directed, un/weighted, multigraph)
\item Undirected graphs: social networks, communication networks (duplex communication)
\item Directed graphs: hyperlinks, (class, module, package) dependencies,
logical circuits, job scheduling, flow graphs
\item Edge-weighted graphs: roadmaps with geographical distance, or with toll,
communication networks with bandwidth
\item Edge-weighted digraphs: job scheduling with duration, flow with volume,
financial transactions with size
\end{enumerate}
\end{frame}

\begin{frame}
    \frametitle{Undirected Graphs}

\begin{itemize}[<+->]
\item Undirected graph: a set of \emph{vertices} (or \emph{nodes}) $V$
and a set of \emph{edges} $E$ connecting the nodes
\item\emph{Subgraph}: subset of $E$ and subset of $V$ forming a graph (!)
\item\emph{Path}: sequence of nodes connected by edges (!)
\item\emph{Simple path}: path with no node repeated
\item\emph{Length of path}: number of edges
%\item\emph{Distance} between  two nodes: length of a shortest connecting path
%if there is a path connecting these nodes, otherwise $\infty$
\item\emph{Cycle}: path of length $>0$ with same start and end node
\item\emph{Simple} cycle: not repeating edges or nodes
(apart from start and end node)
\item\emph{Acyclic graph}: graph without simple cycles
\item\emph{Connected graph}: having a path between every two nodes
\item\emph{Connected component}: a maximal connected subgraph
\end{itemize}
\end{frame}

\begin{frame}\label{tree_forest}
    \frametitle{Trees and Forests}

\begin{itemize}[<+->]
\item Example: {\tt tinyG.txt} on bb
\item `Anomalies' concerning edges:
  \begin{itemize}
  \item Self-loop: edge connecting a node to itself
  \item Parallel edges: two edges connecting the same node(s)
  \end{itemize}
\item When no anomalies, $E \subseteq \set{\set{v,v'}\mid v\in V, v'\in V, v\neq v'}$
\item\emph{Tree}: connected acyclic graph (then: no anomalies)
\item\emph{Spanning tree}: maximal subgraph that is a tree
\item Lemma: any spanning tree of a connected graph contains all nodes.
Proof on bb.
\item\emph{Forest}: graph consisting of disjoint trees
%\item Lemma: any connected component of a forest is a tree
\item\emph{Spanning Forest}: forest consisting of spanning trees of connected components of a graph
\end{itemize}
\end{frame}

\begin{frame}\label{tree_characteristics}
    \frametitle{Undirected Graphs (ctnd)}

\begin{itemize}[<+->]
\item\emph{Distance} between  two nodes: length of a shortest connecting path
if there is a path connecting these nodes, otherwise $\infty$
\item\emph{Degree} of a node: number of edges connected to that node
\item Graph $G=(V,E)$, the following are equivalent:
  \begin{itemize}
  \item $G$ is a tree (def: connected and acyclic)
  \item $G$ has $|V|-1$ edges and no cycles  
  \item $G$ has $|V|-1$ edges and is connected
  \item $G$ is acyclic and adding an edge creates a cycle
  \item Any two nodes of $G$ are connected by exactly one simple path
  \end{itemize}
\item Example: some (connected) subgraphs of {\tt tinyG.txt}
$\xymatrix@R=1cm@C=1cm{
1 & 0\ar@{-}[l]\ar@{-}[d]\ar@{-}[r]\ar@{-}@(dr,dl)[rr] &2 & 6 & 
7 \ar@{-}[d]& 9 \ar@{-}[d]\ar@{-}[rd] \ar@{-}[r]& 10\\
   & 5 \ar@{-}@(ur,ul)[rr] \ar@{-}[r] & 3  \ar@{-}[r] & 4   \ar@{-}[u] & 
8                   &  11\ar@{-}[r] & 12 }$
\end{itemize}
\end{frame}

\begin{frame}
    \frametitle{Graph representation and implementation}

\begin{itemize}[<+->]
\item Impractical: 
\href{https://en.wikipedia.org/wiki/Adjacency_matrix}%
{\color{red}{adjacency matrix}} ${\sim}V^2$, 
\href{https://en.wikipedia.org/wiki/Incidence_matrix}%
{\color{red}{incidence matrix}}  ${\sim}VE$
\item Often practical: 
\href{https://en.wikipedia.org/wiki/Adjacency_list}%
{\color{red}{adjacency lists}}  $\sim(V{+}2E)$, that is, 
%{\tt LinkedList<Integer>[] adj},
{\tt adj[v]} lists all nodes {\tt w} connected to {\tt v} by an edge
\item Example: {\tt tinyG.txt} by 
\href{\git/programs/graphs/LinkedListG.java}%
{\color{red}{\tt LinkedListG.java}}
\item Graph API includes: {\tt V(), E(), addEdge()}
\item Basic algorithms: Depth-First Search (DFS) and Breadth-First Search (BFS)
\item Both DFS and BFS `walk through the graph', but differently
\item Both DFS and BFS can compute a spanning tree and forest
\end{itemize}
\end{frame}

%// dfs() is recursive, call: dfs(v,marked);
%// bfs() not recursive, call: q.enqueue(v); bfs(q,marked); 

\begin{verbatim}
public void dfs(Integer v, boolean[] marked) {
  marked[v] = true;
  for (Integer w : adj[v])
    if (! marked[w]) dfs(w,marked);
} // dfs() is recursive, call: dfs(v,marked);

public void bfs(Queue<Integer> q, boolean[] marked) {
  while (!q.isEmpty()) {
     Integer v = q.dequeue();
     for (Integer w : adj[v]) 
       if (! marked[w]) {marked[w]=true; q.enqueue(w);}
  }
} // call: marked[v]=true; q.enqueue(v); bfs(q,marked);
 
// Example: 0-1, 0-3, 1-2, 1-3, 3-4
// Example: complete ternary tree of height 2
\end{verbatim}

\begin{frame}
    \frametitle{Implementation and Properties of DFS/BFS}

\begin{itemize}[<+->]
\item \href{\git/programs/graphs/LinkedListG.java}% 
{\color{red}{LinkedListG.java}}: {\tt pathdfs(), pathbfs()}
\item DFS and BFS mark nodes connected to a given source node
in time proportional to the sum of their degrees ($\leq 2E$),
and can return a path from a marked node to the given source
in time proportional to the length of this path
\item BFS always finds a shortest path 
(proof: queue only contains nodes at distance $k$ 
followed by nodes at distance $k+1$, 
while all nodes at distance $\leq k$ not in queue have been processed)
\item DFS finds a left-most path (long or short, example bb)
\item BFS tends to use more space (but not always)
\item UF (from Ch. 1) tests connectivity, but finds no paths
\end{itemize}
\end{frame}

\begin{frame}
    \frametitle{Applications}

\begin{itemize}[<+->]
\item \href{\git/programs/graphs/StringSTG.java}% 
{\color{red}{StringSTG.java}}, flight connections, shortest path = minimum number of
stop-overs
\item Degrees of separation in social networks, e.g., 
Erd\"{o}s number = length of shortest path to Paul Erd\"{o}s in the co-author graph
\item Connected components:
\href{\git/programs/graphs/LinkedListG.java}%
{\color{red}{\tt LinkedListG.countcc()}} 
\item Example: {\tt tinyG.txt} has three connected components
\[
\xymatrix@R=1cm@C=1cm{
1 & 0\ar@{-}[l]\ar@{-}[d]\ar@{-}[r]\ar@{-}@(dr,dl)[rr] &2 & 6 & 
7 \ar@{-}[d]& 9 \ar@{-}[d]\ar@{-}[rd] \ar@{-}[r]& 10\\
   & 5 \ar@{-}@(ur,ul)[rr] \ar@{-}[r] & 3  \ar@{-}[r] & 4   \ar@{-}[u] & 
8                   &  11\ar@{-}[r] & 12 }
\]
\end{itemize}
\end{frame}


\section{Ch.4.2 Directed Graphs}

\begin{frame}
    \frametitle{Directed Graphs}

\begin{itemize}[<+->]
\item\emph{Digraph}: a set of \emph{vertices} (or \emph{nodes}) $V$
and a set of \emph{directed edges} (or \emph{arrows}) $E$ pointing from one node to another
\item\emph{Subdigraph, directed (simple) path (\emph{{\color{red}di}path}), 
directed (simple) cycle, acyclic, length}: as expected
\item Often we leave out `di' in digraph, dipath, etc.
\item\emph{DAG}: {\color{red}D}irected {\color{red}A}cyclic {\color{red}G}raph;
%\emph{{\color{red}di}path}: {\color{red}di}rected path
\item\emph{Degree}: {\color{red}in}-degree and {\color{red}out}-degree
\item Node $v$ is \emph{reachable} from $w$: a dipath  from $w$ to $v$ exists
\item\emph{Strongly connected digraph}: dipath between every two nodes (for all
$v,w$, there are dipaths from $v$ to $w$ and from $w$ to $v$)
\item\emph{Strongly connected component}: maximal strongly connected subgraph
($u\rightleftarrows v \to w$ has two scc's)
\item Representation: adjacency lists even simpler!
\end{itemize}
\end{frame}

\begin{frame}
    \frametitle{Directed Graph, example}

{\tt tinyCG.txt}:

\[
\xymatrix@R=1cm@C=1cm{
0 \ar[d] \ar[r] \ar@(ur,ul)[rr]& 1\ar[ld] &5 \\
2 \ar[r] \ar@(dr,dl)[rr]  & 3 \ar[r] \ar[ur]\ar[r] & 4 
}
\]
\end{frame}

\begin{frame}
    \frametitle{Reachability Problems}
    Assume we are given a directed graph $G$.

\begin{itemize}[<+->]

\item Single-source: given a node $s$, the \emph{source}, 
is a given node $v$ reachable from $s$? Example: {\tt tinyCG.txt}
\item Multiple-source: given a set of nodes $S$, is a given node $v$ reachable from some node in $S$?
\item Solutions: same DFS and BFS algorithms as in Chapter 1
\item Application (example): 
\href{https://en.wikipedia.org/wiki/Tracing_garbage_collection}%
{\color{red}{mark-and-sweep garbage collection}}
\item Single-source path:  given nodes $s,v$ such that $v$ is reachable from $s$.
      Find a path from $s$ to $v$.
\item Single-source shortest path: given nodes $s,v$ such that $v$ is reachable from $s$.
      Find a \emph{shortest} path from $s$ to $v$.
\item Solutions: same DFS (path) and BFS (shortest path) algorithms as for undirected graphs
\end{itemize}
\end{frame}

\begin{frame}
    \frametitle{Cycle Detection}
   % Assume we are given a directed graph $G$.

\begin{itemize}[<+->]

\item Recall: a \emph{DAG} is a graph without a directed cycle
\item Acyclicity test, cycle detection: easy extension of DFS. We keep track of the
search path from the source. If there is an arrow from $v$ to $w$ and $w$ is on the path
from the source to $v$, then there is a cycle. (DFS finds the leftmost path to the leftmost
cycle.) Two techniques (space-time trade-off!):
  \begin{itemize}
  \item Go back the search path:
  \href{\git/programs/graphs/directedGraphs/LinkedListDiG.java}%
{\color{red}{\tt LinkedListDiG.slowCyclist()}}
  \item Memorize the search path:
  \href{\git/programs/graphs/directedGraphs/LinkedListDiG.java}%
{\color{red}{\tt LinkedListDiG.fastCyclist()}} 
  \end{itemize}
\item Application: precedence scheduling of jobs
\item Example: {\tt cycleG.txt}
\[
\xymatrix@R=1cm@C=1cm{
0\ar[d]\ar[r]& 2\ar@(dl,ul)[d]\\
1  & 3 \ar@(ur,dr)[u] 
}
\]
\end{itemize}
\end{frame}


\begin{frame}
    \frametitle{Pre-order, post-order, in-order}

\begin{itemize}[<+->]
\item Graph walks based on DFS from a source node
\item Pre-order: order in which DFS arrives at nodes
\item Post-order: order in which DFS leaves nodes
\item In-order \emph{for binary trees}: e.g., in
\href{\git/programs/searching/elementarySymbolTables/UBST.java}%
{\color{red}{\tt UBST.show()}}
\item Example:
\[\xymatrix@R=.5cm@C=.1cm{
\text{pre-order:~}           01234\quad&&&&0 \ar@{>}[ld]\ar@{>}[rd]&&\\
\text{post-order:~}          23140&&&1\ar@{>}[ld]\ar@{>}[rd]&&4\\
%\text{reversed post-order:~}0312&&2}
\text{in-order:~}            21304&&2&&3}
\]
\item Example: {\tt tinyCG.txt} on bb and by 
\href{\git/programs/graphs/directedGraphs/LinkedListDiG.java}%
{\color{red}{\tt LinkedListDiG.java}}
\end{itemize}
\end{frame}

\begin{frame}
    \frametitle{Topological order of acyclic digraph}

\begin{itemize}[<+->]
\item Topological order: total order $\prec$ 
compatible with the graph in the following sense:
if there is an arrow from $u$ to $v$, then $v\prec u$ 
(consequently: $v\preceq u$ if $v$ is reachable from $u$)
\item NB one can also take $\succ$, this is only a matter of definition
\item Lemma: if a digraph has a topological order, then it is acyclic
(proof: a cycle cannot be ordered compatibly)
\item Lemma: if a digraph is acyclic, then it has a topological order
(proof idea: if acyclic, the post-order is a topological order since,
if there is an arrow from $u$ to $v$, then $u$ is not reachable from $v$
and DFS will leave/visit $u$ after it has left $v$)
\item Topological order is a job schedule respecting precedence
\item Many topological orderings of the acyclic graph\\  {\tt 1 <- 2 <- 4 -> 5 -> 3 <- 0 }
\end{itemize}
\end{frame}

\begin{frame}[fragile]
    \frametitle{Transitive closure}

\begin{itemize}[<+->]
\item Definition: given $G$, its (reflexive!) transitive closure $G^*$ is a 
graph with the same nodes, with arrows from $u$ to $v$ for each $v$
that is reachable from $u$ in $G$.
\item NB: $G^*$ can have many more arrows than $G$
\item Implementation: adjacency matrix in case of many arrows
\begin{verbatim}
...
boolean[][] adjmat = new boolean[V][V];
for (int v=0; v<V; v++) {
   boolean[] marked = new boolean[V];
   dfs(v,marked);
   adjmat[v] = marked; // adjmat[v][v]==true: reflexive
...
 }    
\end{verbatim}  
\end{itemize}
\end{frame}

\section{Ch.4.3 Minimum Spanning Tree}

\begin{frame}
    \frametitle{Minimum Spanning Tree}

\begin{itemize}[<+->]
\item Recall slide \hyperlink{tree_forest}{\color{red}{\pageref{tree_forest}}}: 
\emph{spanning tree} of a connected undirected graph
is a maximal subgraph that is a tree (and thus contains all nodes and is acyclic)
\item EWG = Edge-Weighted Graph, here always connected
%\item Example: {\tt tinyEWG.txt} on bb
\item Recall slide \hyperlink{tree_characteristics}{\color{red}{\pageref{tree_characteristics}}}:
all spanning trees have $V-1$ edges
\item \emph{Weight} of spanning tree: sum of the weights of its edges
\item \emph{Minimum Spanning Tree}: spanning tree with minimal weigth
\item Example: three MSTs of $\xymatrix@R=.5cm@C=.1cm{
&0 \ar@{-}[ld]_1\ar@{-}[rd]^1&\\
1 \ar@{-}[rr]_1&&2}$
\item Exc.4.3.3: if all weights different, then MST is unique
\item From now on we assume all weights different!
\end{itemize} 
\end{frame}

\begin{frame}
    \frametitle{MST Example (tinyEWG.txt)}
\[
\xymatrix@R=0.7cm@C=2cm{
0 \ar@{-} `u/10pt[rrr] `/10pt[rrr] [rrddddd]^{0.16}
\ar@{-}[ddddd]_{0.38}\ar@{-}[rr]_{0.26}&&2\ar@{-}[ddddd]_{0.34}&\\
  &6\ar@{-}[ru]^{0.4}\ar@{-}[lu]^{0.58}&&\\
  &3\ar@{-}[u]|{0.52}\ar@{-}[ruu]|{0.17}&&\\
  &1\ar@{-}[u]|{0.29}\ar@{-}[rdd]|{0.19}\ar@{-}[ruuu]|{0.36}&&\\
  &5\ar@{-}[rd]_{0.28}\ar@{-}[u]|{0.32}&&\\
4\ar@{-}[rr]^{0.37}\ar@{-}[ru]^{0.35}\ar@{-}[ruuuu]^{0.93}&& 7&\\
}
\]
\end{frame}

\begin{frame}
    \frametitle{Minimum Spanning Tree (ctnd)}

\begin{itemize}[<+->]
\item Applications: power plants and electrical grid, 
airlines and flight routes, maps and distance
\item Weights may be zero or negative (e.g., cost minus profit of a new network of roads between cities)
\item Two important algorithms to find the MST: Prim's and Kruskal's
\item We explain Kruskal's algorithm first (unlike the book), since in previous years
students found Prim's algorithm more difficult to understand
\end{itemize} 
\end{frame}

\begin{frame}
    \frametitle{Cuts and Crossing Edges}

\begin{itemize}[<+->]
\item Recall slide \hyperlink{tree_characteristics}{\color{red}{\pageref{tree_characteristics}}}:
deleting an edge from a tree creates two disjoint components, adding an edge creates a cycle
\item \emph{Cut}: a partition of $V$ in two non-empty subsets of nodes 
\item \emph{Crossing edge}: edge connecting two nodes in different subsets of a cut

\item NB there can be more than one crossing edge: $\xymatrix@R=.5cm@C=.5cm{
0\ar@{-}[r]&2 \ar@{.}[d]\\1\ar@{-}[r] \ar@{.}[ru] &3 }$
\item Lemma: for any cut in an EWG, the crossing edge of minimum weight is in the MST.
\item Proof: given a cut, assume by contradiction
there is a crossing edge $e$ of weight smaller than the crossing
edge(s) that is (are) in the MST (e.g., the dotted edges above). Adding $e$ creates a simple cycle,
which must contain one other crossing edge $f$ in the MST. Replacing $f$ by $e$: \Lightning
\end{itemize} 
\end{frame}

\begin{frame}
    \frametitle{Kruskal's Algorithm}

\begin{itemize}[<+->]
\item Datastructures:
  \begin{itemize}
  \item EWG represented with adjacency lists {\tt adj[v]}
  \item Minimum priority queue {\tt pq} for edges
  \item Union-Find object {\tt uf} testing connectivity
  \item Queue {\tt mst} for the minimum spanning tree
\end{itemize}
\item Algorithm:
  \begin{enumerate}
  \item delete the minimum edge {\tt e} from {\tt pq}
  \item if the points connected by {\tt e} are not connected, 
  add {\tt e} to {\tt mst} and connect the points in {\tt uf}
  \item continue at point 1 until {\tt pq} is empty or {\tt uf} contains all nodes
  \end{enumerate}
  \item Examples: EWG on slide 
  \hyperlink{EWGexample}{\color{red}{\pageref{EWGexample}}}, {\tt tinyEWD.txt}
  \item Correctness: by lemma about minimum-weight crossing edge of cut
  (NB it is also correct to use a sorted array of edges)
  \item Implementation: \href{http://algs4.cs.princeton.edu/code/edu/princeton/cs/algs4/KruskalMST.java.html}%
{\color{red}{KruskalMST.java}}, constructor method
\end{itemize} 
\end{frame}


\begin{frame}
    \frametitle{Prim's Lazy Algorithm}

\begin{itemize}[<+->]
\item Datastructures:
  \begin{itemize}
  \item EWG represented with adjacency lists {\tt adj[v]}
  \item Minimum priority queue {\tt pq} for edges
  \item Array {\tt marked[v]} for marking vertices
  \item Queue {\tt mst} for the minimum spanning tree
\end{itemize}
\item Edge is \emph{eligible} if not both endpoints marked (crossing!)
\item Algorithm based on previous lemma, cut: un/marked nodes

  \begin{enumerate}
  \item mark $0$ and add all eligible edges in {\tt adj[0]} to {\tt pq}
  \item as long as {\tt pq} is not empty, do:
\begin{enumerate}
  \item get and delete minimum edge {\tt e} from {\tt pq}
  \item add {\tt e} to {\tt mst}, say the unmarked endpoint of {\tt e} is {\tt k}
  \item mark {\tt k} and add all eligible edges in {\tt adj[k]} to {\tt pq}
  \item delete ineligible minimum edges from {\tt pq} 
\end{enumerate}
  \end{enumerate}
\item After this algorithm, the queue {\tt mst} contains the MST
\end{itemize} 
\end{frame}

\begin{frame}
    \frametitle{Prim's Algorithm (ctnd)}

\begin{itemize}[<+->]
\item \href{http://algs4.cs.princeton.edu/code/edu/princeton/cs/algs4/LazyPrimMST.java.html}%
{\color{red}{LazyPrimMST.java}}, methods {\tt scan()} and {\tt prim()}
\item Invariant: at least one of the nodes of an edge in {\tt pq} is marked
\item NOT: all edges in {\tt pq} are crossing edges wrt cut un/marked
\item Lazy: ineligible edges are not eagerly deleted from {\tt pq}
\item Runtime: LazyPrimMST runs in $O(E\log E)$ time (worst-case)
\item Possible: \emph{only} crossing edges wrt cut un/marked  in {\tt pq}
\item Eager: if {\tt v} unmarked, the only crossing edge of interest is the
\emph{lightest} one connecting {\tt v} to the marked edges (= MST so far)
\item Runtime: eager Prim runs in $O(E\log V)$ time (worst-case)
\item Max size {\tt pq}: $E$ edges for lazy; $V$ nodes for eager
\end{itemize} 
\end{frame}

\begin{frame}\label{EWGexample}
    \frametitle{Prim's Eager Algorithm}

\begin{itemize}[<+->]
\item Datastructures:
  \begin{itemize}
  \item EWG represented with adjacency lists {\tt adj[v]}
  \item Boolean array {\tt marked[v]} for marking vertices
  \item Array {\tt distTo[v]}, minimum distances to MST so far
  \item Array {\tt edgeTo[v]}, edges with minimum distance to MST so far
  \item Indexed minimum priority queue {\tt pq}: index={\tt v}, key={\tt distTo[v]}
  \item Queue {\tt mst} for the MST  based on {\tt edgeTo}
\end{itemize}
\item Example: $\xymatrix@R=1.5cm@C=1.5cm{
0\ar@{-}[r]^2 \ar@{-}[d]_1 \ar@{-}[rd]^3 &2 \ar@{-}[d]^{-1}\\1\ar@{-}[r]_0  &3 }$
\item \href{http://algs4.cs.princeton.edu/code/edu/princeton/cs/algs4/PrimMST.java.html}%
{\color{red}{PrimMST.java}}, methods {\tt scan()} and {\tt prim()}
\end{itemize} 
\end{frame}


\begin{frame}
    \frametitle{Memory-Use and Run-time Analysis}
\begin{itemize}[<+->] \item Space, worst-case:
\begin{itemize}[<+->]
\item All methods use $O(V+E)$ space for the graph, plus ...
\item Priority queue for edges (Lazy Prim and Kruskal): $O(E)$ space
\item Priority queue for vertices (Eager Prim): $O(V)$ space
\item Arrays indexed by vertices (all): $O(V)$ space
\end{itemize} 
\item Time, worst-case:
\begin{itemize}[<+->]
\item Priority queue operations (Lazy Prim and Kruskal): $O(E\log E)$ time
\item Priority queue operations (Eager Prim): $O(E\log V)$ time
\end{itemize}
\item NB: $E\leq V^2$ implies $\log E \leq \log V^2 \leq 2\log V$
\item Example: complete graph on $0,\ldots,9$, edge $n$-$m$ weight $n{+}m$,
MST consists of $0$-$m$ for $m=1,\ldots,9$
\end{itemize} 
\end{frame}

\section{Ch.4.4 Shortest Paths}

\begin{frame}
    \frametitle{Job scheduling --- precedence}\label{Job_prec}

\begin{itemize}[<+->]
\item Example: assume we have jobs $0,\ldots,5$ and job 0 must be done before we can
do 2 and 5, job 1 must be done between job 0 and 2, and 3 between 2 and 4,
and 5 and 4 require 3 to be done.
\item Sounds familiar? {\tt tinyCG.txt}!
\item Can we find such a schedule?\\Yes, e.g., 012345
\item What if also job 5 must precede job 4? (Yes)
\item What if also job 5 must precede job 2? (No)
\item Any (reversed) topological order gives a correct schedule
\item A topological order exists iff the graph is acyclic
\end{itemize} 
\end{frame}

\begin{frame}
    \frametitle{Job scheduling --- precedence and duration}\label{Job_dur}

\begin{itemize}[<+->]

\item In the previous example we have abstracted time to precedence
\item What if jobs also have duration and we want the fastest schedule?
\item Example: {\tt tinyCG.txt}, with all jobs one hour, and two workers
\item Fastest schedule takes five hours (why?)
\item The longest path is the shortest path when we negate all weights
\item More examples later: \href{\git/programs/tinyJob.txt}%
{\color{red}{\tt tinyJob.txt}}, \href{\git/programs/tiNoJob.txt}%
{\color{red}{\tt tiNoJob.txt}}
\end{itemize} 
\end{frame}


\begin{frame}
    \frametitle{Shortest paths}\label{SPintro}

\begin{itemize}[<+->]
\item Recall slide \hyperlink{graph_classes}{\color{red}{\pageref{graph_classes}}}, 
\emph{Edge-weighted digraphs}: digraphs  in which every arrow has a weight
\item \emph{Weight} of a (di)path: sum of weights of the arrows
\item \emph{Shortest} path from node $s$ to node $t$: minimum path weight
\item EWD = Edge-Weighted Digraph, example: {\tt tinyEWD.txt}
\item Example: two SPs from 0 to 3: $\xymatrix@R=1.5cm@C=1.5cm{
0\ar@{>}[r]|1 \ar@/^/[rd]|8 &2\ar@/^/[ld]|2 \ar@{>}[d]|5\\1\ar@{>}[r]|3  &3 }$
\item Shortest paths need not be unique, even if all weights are different!
\item Shortest paths need not exist, for two independent reasons:
  \begin{itemize}[<+->]
  \item When target $t$ is not reachable from source $s$
  \item When there is a negative cycle on the path to $t$,
        e.g., $\xymatrix@R=0.5cm@C=0.5cm{1\ar@(ur,ul)}^{-1}$
  \end{itemize}
\end{itemize} 
\end{frame}

\begin{frame}
    \frametitle{Variations}

\begin{itemize}[<+->]
\item Single-source versus multiple sources
\item Only non-negative weights versus all weights allowed
\item Acyclic versus cycles, in particular negative cycles
\item Longest path: shortest path with weights negated
\item Simple example: scheduling of jobs A, B, and C
  \begin{itemize}[<+->]
  \item A (3 hrs), must precede by B (2 hr), independent C (4 hrs)
  \item Graph: $\xymatrix@R=0.5cm@C=0.5cm{
  &As\ar[r]^3&Ae\ar[r]^0\ar@{.>}[rrrd]&Bs\ar[r]^2&Be\ar@{.>}[rd]\\
  s\ar@{.>}[ru]\ar@{.>}[rd]\ar@{.>}[rrru]&&&&&t\\
  &Cs\ar[r]_4&Ce\ar@{.>}[rrru]}$
  \item Arrow from $X$ to $Y$ with weight $d$ means $time_Y \geq time_X +d$ 
  ($d$ can be negative!)
  \item Now add: $As$ after 2 hrs before $Bs$, 
  $time_{As} \geq time_{Bs} -2$. Feasible? (No)
  And $As$ exactly 4 hrs before $Bs$? (Yes)
  \end{itemize}
\end{itemize} 
\end{frame}


\begin{frame}
    \frametitle{Dijkstra's Algorithm}

\begin{itemize}[<+->]
\item Single-source, only non-negative weights, cycles no problem
\item Datastructures:
  \begin{itemize}
  \item EWD with adjacency lists {\tt adj[v]} of weighted out-edges
  \item Boolean array {\tt marked[v]} for marking vertices
  \item Array {\tt distToSource[v]}, minimum distances to source so far
  \item Array {\tt pathToSource[v]}, best arrow to source so far
\end{itemize}
\item Algorithm: relax (see below) and mark the unmarked node with least distance, until all marked;
Simple example: slide \hyperlink{SPintro}{\color{red}{\pageref{SPintro}}}
\item Invariants:
\begin{itemize}[<+->]
  \item Marked nodes: known shortest path to source (non-negativity!)
  \item Unmarked nodes: known shortest path to source THROUGH marked nodes 
  (requires in-arrow from marked node)
  \end{itemize}
\item Implementation \href{\git/programs/graphs/LinkedListEWD.java}%
{\color{red}{\tt LinkedListEWD.slowEWD()}}, examples \href{\git/programs/myEWD.txt}%
{\color{red}{\tt myEWD.txt}} and {\tt tinyEWD.txt}
\end{itemize} 
\end{frame}

\begin{frame}
    \frametitle{Relaxation}
\begin{itemize}[<+->] 
\item Assume an  array  {\tt distToSource[v]} with
minimum distances, so far, to a given source
\item To \emph{relax an edge e} from $u$ to $v$ with weight $x$ means
to update {\tt distToSource[v]} to {\tt distToSource[u]}$+x$ if the latter is smaller
\item To \emph{relax a node u} means to relax all edges in {\tt adj[u]}, that is,
to update {\tt distToSource[v]} to {\tt distToSource[u]}$+x$ if the latter is smaller,
for every edge from $u$ to $v$ with weight $x$
\item Dijkstra: relax and mark unmarked node $v$ with minimal {\tt distToSource[v]},
until all nodes marked
\item Bellman-Ford: do max $V$ rounds of relaxation of all edges
(may stop after a round without updates)
\end{itemize} 
\end{frame}


\begin{frame}
    \frametitle{EWD Example (tinyEWD.txt, NB $4\leftrightarrow 5\leftrightarrow 7$!)}
\[
\xymatrix@R=0.7cm@C=2cm{
0\ar[ddddd]_{0.38}\ar[rr]^{0.26}&&2\ar[ddddd]^{0.34}&\\
  &6\ar[ru]^{0.4}\ar[lu]^{0.58}\ar[ldddd]_{0.93}&&\\
  &3\ar[u]|{0.52}&&\\
  &1\ar[u]|{0.29}&&\\
  &5\ar[u]|{0.32}&&\\
4\ar[rr]_{0.37}\ar@{<->}[ru]^{0.35}&& 7\ar@{<->}[lu]^{0.28}\ar[luuu]|{0.39}&\\
}
\]
\end{frame}


\begin{frame}
    \frametitle{Bellman-Ford}

\begin{itemize}[<+->]
\item Single-source, also negative weights, negative cycles detected
\item Datastructures:
  \begin{itemize}
  \item EWD with adjacency lists {\tt adj[v]} of weighted out-edges
  \item Array {\tt distToSource[v]}, minimum distances to source so far
  \item (Array {\tt pathToSource[v]}, best arrow to source so far)
\end{itemize}
\item Algorithm: do at most $V$ rounds for every node {\tt v} 
and every arrow {\tt e} in {\tt adj[v]}, if {\tt e} shortens the
distance to its endpoint {\tt w}, update that distance (and path);
stop after a round when no distances improve. If distances improve in the
$V$-th round, a negative cycle is reachable from the source.
\item Invariant: after $n$ rounds the distances are less than or
equal to the shortest path of length $n$ from the source
\item Implementation 
\href{\git/programs/graphs/LinkedListEWD.java}%
{\color{red}{\tt LinkedListEWD.simpleBF()}}, examples 
\href{\git/programs/myEWD.txt}%
{\color{red}{\tt myEWD.txt}},
\href{\git/programs/tinyJob.txt}%
{\color{red}{\tt tinyJob.txt}},
\href{\git/programs/tiNoJob.txt}%
{\color{red}{\tt tiNoJob.txt}} and {\tt tinyEWD.txt}
\end{itemize} 
\end{frame}

\begin{frame}
    \frametitle{Memory-Use and Run-time Analysis}
\begin{itemize}[<+->] \item Space
\begin{itemize}[<+->]
\item All methods use $O(V+E)$ for the graph, plus $O(V)$ extra 
\item Still true for Dijkstra improved with an indexed priority queue,
but indexed priority queue takes $\sim 3V$ space
\end{itemize} 
\item Time, worst-case:
\begin{itemize}[<+->]
\item $V$ times finding a minimum (original Dijkstra): $O(V^2)$ 
\item Priority queue operations (improved Dijkstra): $O(E\log V)$
%(best way to see this: every edge is processed once, using an IPQ with nodes)
\item $V$ rounds relaxing $E$ edges (Bellman-Ford): $O(EV)$ 
\end{itemize}
\end{itemize} 
\end{frame}



\section{Odds and Ends Chapter 4}

\begin{frame}
    \frametitle{Odds and Ends Chapter 4}

\begin{itemize}[<+->]
%\item 
\item Indexed Priority Queue
\item Eager Prim MST Algorithm
\end{itemize}
\end{frame}


\section{Table of Contents}

\begin{frame}%[fragile]
    
    \frametitle{ToC and topics of general interest}

 \begin{itemize}[<+->]\label{contentstable}
    \item Table of Contents on next slide (all items clickable)
    \item Practical stuff: slide \hyperlink{practicalstuff}{\color{red}{\pageref{practicalstuff}}}
 \end{itemize}

\end{frame}


\begin{multicols}{2}
\tableofcontents %cannot be a frame: bug in beamer ?
\end{multicols}

\end{document}
